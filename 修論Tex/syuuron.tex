%%%%%%%%%%%%%%%%%%%%����%%%%%%%%%%%%%%%%%%%%
\documentclass[a4paper,12pt]{jarticle}	
%!TEX encoding = UTF-8 Unicode

%%%%%%%%%%%%%%%%%%%%�p�b�P�[�W�̓ǂݍ���%%%%%%%%%%%%%%%%%%%%
\usepackage{amsmath}
\usepackage{enumerate}
\usepackage{bm}
\usepackage[dvipdfmx]{graphicx}
\usepackage{ascmac}
\usepackage{amsthm}
\usepackage{bmpsize}
\usepackage{subfig}
\usepackage{nidanfloat}
\usepackage{here}
\usepackage{comment}
\usepackage{url}
\usepackage{amsmath} 			
\usepackage{amssymb}			
\usepackage{mathrsfs}
\usepackage[titletoc, title]{appendix}
%\usepackage{indentfirst} 			
%%%%%%%%%%%%%%%%%%%%�}�[�W���̐ݒ�%%%%%%%%%%%%%%%%%%%%
\topmargin=10mm 	
\voffset=-1in 		
\headheight=5mm	  	
\headsep=10mm 		
\textheight=252mm 	
\topskip=0mm 		
\footskip=10mm  	

\evensidemargin=30mm 	
\oddsidemargin=30mm 	
\hoffset=-1in 			
\textwidth=170mm 		

\marginparpush=0mm 		
\marginparsep=0mm 		
\marginparwidth=0mm 	

%%%%%%%%%%%%%%%%%%%%�T�v�C�}�C�\�Ȃǂ̖��O���ς���%%%%%%%%%%%%%%%%%%%%
\renewcommand{\thesection}{\arabic{section}.}
\renewcommand{\thesubsection}{\arabic{section}.\arabic{subsection}}
\renewcommand{\figurename}{{Fig}.}
\renewcommand{\tablename}{{Table}}
\renewcommand{\contentsname}{目次}
\renewcommand{\appendixname}{Appendix}
\renewcommand{\refname}{参考文献}
\renewcommand{\labelenumi}{(\Roman{enumi})}


\newcommand{\eref}[1]{Eq.~(\ref{#1})}
\newcommand{\esref}[2]{Eqs.~(\ref{#1}) and (\ref{#2})}
\newcommand{\essref}[2]{Eqs.~(\ref{#1})$\sim$(\ref{#2})}
\newcommand{\figref}[1]{Fig.~\ref{#1}}
\newcommand{\figsref}[2]{Figs.~\ref{#1} and \ref{#2}}
\newcommand{\figssref}[2]{Figs.~\ref{#1}$\sim$\ref{#2}}
\newcommand{\tabref}[1]{Table \ref{#1}}
\newcommand{\secref}[1]{\S \ \ref{#1}}

\makeatletter
\renewcommand{\@cite}[1]{\textsuperscript{(#1)}}
\renewcommand{\@biblabel}[1]{(#1)}
\makeatother

\newcommand{\citeref}[1]{\textsuperscript{\cite{#1}}}
\newcommand{\citesref}[2]{\textsuperscript{\cite{#1},~\cite{#2}}}
\newcommand{\citessref}[2]{\textsuperscript{\cite{#1}$\sim$\cite{#2}}}

\begin{document}
\input{dummy}
\baselineskip=18pt 	

\thispagestyle{empty} 		
%\pagenumbering{roman}	
%\setcounter{page}{0} 		
\pagestyle{plain} 		

\makeatletter
\def\@maketitle{%
\begin{center}%
\let\footnote\thanks
\vspace*{25mm} % 
{\LARGE \@title \par}%
\vskip 1.5em%
{\large
\lineskip .5em%
\begin{tabular}[t]{c}%
\@author
\end{tabular}\par}%
\vskip 1em%
{\large \@date}%
\end{center}%
\par\vskip 1.5em}
\makeatother 

\makeatletter
\long\def\@makecaption#1#2{%
\vskip\abovecaptionskip
\iftdir\sbox\@tempboxa{#1\hskip1zw#2}%
\else\sbox\@tempboxa{#1 #2}%
\fi
\ifdim \wd\@tempboxa >\hsize
\iftdir #1\hskip1zw#2\relax\par
\else #1 #2\relax\par\fi
\else
\global \@minipagefalse
\hbox to\hsize{\hfil\box\@tempboxa\hfil}%
\fi
\vskip\belowcaptionskip}
\makeatother




%%%%%%%%%%%%%%%%%%%%%%%%%%%%%%%%%%%%%%%%%%%%%%%%%%%%%%%%%%%%%%%%%%%%%%%%%%%%%%%%%%%%%%%%%%
%%%%%%%%%%%%%%%%%%%%%%%%%%%%%%%%%%%%%%%% Abstract %%%%%%%%%%%%%%%%%%%%%%%%%%%%%%%%%%%%%%%%
%%%%%%%%%%%%%%%%%%%%%%%%%%%%%%%%%%%%%%%%%%%%%%%%%%%%%%%%%%%%%%%%%%%%%%%%%%%%%%%%%%%%%%%%%%
\section*{Abstract}
In Japan, a lot of bridges built during high economic growth period from the 1950s to the 1970s get older now. In order to safely use old bridges in the future, periodical inspection is necessary. Many bridges are inspected currently by visual inspection and hammering test by humans to detect cracks and loose bolts. However, bridge inspections by humans are difficult due to the number of bridges, risks of accidents for working persons and traffic control during the inspection. Therefore, various UAVs (Unmanned Aerial Vehicles) have been developed to inspect bridges easily and safely. It is hard to introduce manual operation UAVs to inspection because manual operation is difficult for inspectors who do not have control skills. So, it is necessary to develop autonomous UAV in order to promote the introduction of UAVs. Most of the autonomous UAVs are dependent on GPS (Global Positioning System); however, these UAVs cannot be used for bridge inspection because the bridge inspection UAV flies under the bridge. In this study, a position estimation system using UWB (Ultra-Wide Band), IMU (Inertial Measurement Unit) and a distance sensor and flight control system for the inspection UAV aiming at autonomous flight in non-GPS environment were developed.


\newpage
%%%%%%%%%%%%%%%%%%%%%%%%%%%%%%%%%%%%%%%%%%%%%%%%%%%%%%%%%%%%%%%%%%%%%%%%%%%%%%%%%%%%%%%%%%
%%%%%%%%%%%%%%%%%%%%%%%%%%%%%%%%%%%%%%%% Abstract %%%%%%%%%%%%%%%%%%%%%%%%%%%%%%%%%%%%%%%%
%%%%%%%%%%%%%%%%%%%%%%%%%%%%%%%%%%%%%%%%%%%%%%%%%%%%%%%%%%%%%%%%%%%%%%%%%%%%%%%%%%%%%%%%%%
\section*{概要}
現在日本ではインフラの老朽化が進んでおり,特に橋梁は2023年に半数以上が築50 年を迎える.老朽化した橋梁は定期的な点検が必要になるが,点検に必要なコストや点検作業者の不足が深刻な問題になっている.そのため点検作業の簡略化が望まれており,点検用UAV (Unmanned Aerial Vehicle)の開発・導入が進められている.しかしUAVの操縦経験が浅い点検作業者にとって,点検にマニュアル操縦のUAVの導入することは困難である.したがって自律飛行型の点検用UAVの開発を進める必要がある.今ある自律飛行型UAVの多くはGPS (Global Positioning System)に頼って飛行を行うが,橋梁の点検の場合,ロボットが橋梁の下に潜って点検を行うため,GPSを使うことができない.そこで本研究では点検用UAVの非GPS環境での自律飛行を目的として,UWB (Ultra-Wide Band)や慣性センサ,距離センサを組み合わせた自己位置推定法及びIntegral Backstepping制御を使った自律飛行制御手法を提案する.また小型のクアッドロータを用いて実機での提案手法の評価を行った.

\thispagestyle{empty} 
	
\newpage
%%%%%%%%%%%%%%%%%%%%%%%%%%%%%%%%%%%%%%%%%%%%%%%%%%%%%%%%%%%%%%%%%%%%%%%%%%%%%%%%%%%%%%%%%%
%%%%%%%%%%%%%%%%%%%%%%%%%%%%%%%%%%%%%%%% Contects %%%%%%%%%%%%%%%%%%%%%%%%%%%%%%%%%%%%%%%%
%%%%%%%%%%%%%%%%%%%%%%%%%%%%%%%%%%%%%%%%%%%%%%%%%%%%%%%%%%%%%%%%%%%%%%%%%%%%%%%%%%%%%%%%%%
\thispagestyle{empty} 
\tableofcontents 			
\thispagestyle{empty} 	
\pagenumbering{arabic} 	
\setcounter{page}{0} 		
\pagestyle{plain}
 
\newpage
%%%%%%%%%%%%%%%%%%%%%%%%%%%%%%%%%%%%%%%%%%%%%%%%%%%%%%%%%%%%%%%%%%%%%%%%%%%%%%%%%%%%%%%%%%
%%%%%%%%%%%%%%%%%%%%%%%%%%%%%%%%%%%%%% Introduction %%%%%%%%%%%%%%%%%%%%%%%%%%%%%%%%%%%%%%
%%%%%%%%%%%%%%%%%%%%%%%%%%%%%%%%%%%%%%%%%%%%%%%%%%%%%%%%%%%%%%%%%%%%%%%%%%%%%%%%%%%%%%%%%%

\section{緒言}
\subsection{研究背景}
近年の日本において,インフラ,特にその中においても橋梁の老朽化が問題視されている.国土交通省が発表するデータによると,長さ15 mを超える橋梁の数は2013年時点で40万橋存在し,その中でも建設後50年が経過したものの割合は18 \%(7万1000橋)である.こうした老朽化した橋は今後も増加するとされ,10年後には43 \%,20年後には67 \%と,過半数の橋梁が建設後50年を迎えることとなる.適切に管理,検査がなされていない老朽化した橋梁はやはり崩落事故などの危険性をはらんでいる.橋梁の崩落事故はアメリカのミネアポリスのものが有名である.しかし,日本国内においても崩落や崩落事故になりかけた事例がいくつか存在する.2007年の6月三重県にある国道23号の木曽川大橋で,コンクリートの床版に覆われたトラスの斜材が腐食して破断するという事故が起こった.幸いにもこの事故では崩落や死傷者を出すことは免れたが,ずさんな点検のあり方が浮き彫りになった.事故の1年半ほど前には点検が行われたとされているが,腐食の進んだ鋼材の錆の上から塗装を行うなど,常識を逸脱する補修方法が行われているのにも関わらず,それが見過ごされていた.こういった不適切な橋梁の管理,運営には橋梁を所有する地方自治体の財源不足なども背景とされている.国土交通省によると,1橋を近接目視で点検するだけでも50万円の費用がかかるとされている.しかし,市区町村が管理する橋の年間の維持,修繕費は1橋平均でわずか8万円であり,いかに橋梁の点検に掛ける予算が低いのかがわかる.また,橋梁の点検に用いられる一般的な方法は近接目視及び打音検査であり,傷やひび割れの正確な検出には多くの経験値を要する.これらの理由より,低コストかつ簡便に橋梁を点検可能にする新しい技術を開発することが現在の日本において喫緊の課題である.

\begin{figure}[H]
\begin{center}
\includegraphics[height=70mm]{./Fig2/number_of_bridges.png}
\caption{Number of bridges in Japan by construction year}
\label{Number_of_bridges_in_Japan_by_construction_year}
\end{center}
\end{figure}

そこで,着目したのがドローンなどのロボットを用いた橋梁の点検手法である.ここで,ロボットを用いた点検手法は大別して2種類に分けることができる.1つ目は機体に車輪を有し,橋梁の壁面や床面を伝って移動することで点検箇所へアプローチする地上型ロボットである.もう一つはドローンなどの飛行型ロボットでありこれは前者と異なり,空中から点検箇所にアプローチすることが可能である.地上型ロボットとしては高田らが開発したBIREMが挙げられる.このロボットは車輪の先端に磁石を装備しており,吸着対象が磁性体ならば,床や壁を伝いながら点検箇所へ接近し,カメラを用いた目視検査などがすることが可能である.しかし,このロボットは点検箇所まで構造物が地続きである必要があり,橋梁の途中で大きなギャップがあった場合にその先に進めないといったことや,大きな段差があった場合にそれらを乗り越えられないといった問題もある.しかし,飛行型のロボットの場合,大きなギャップや障害物があってもそれらを乗り越えて点検箇所まで近づける.これらの理由より,本研究では飛行型ロボットを開発するに至った.

\begin{figure}[H]
\begin{center}
\includegraphics[height=70mm]{./Fig2/takadabylim.jpg}
\caption{BIREM}
\label{BIREM}
\end{center}
\end{figure}


\subsection{インフラ点検用UAVの開発}
先行研究において,赤堀らはインフラ点検用UAVを開発した[].このUAVは機体の上部にEPM(Electro-Parmanent Magnet)を搭載し,磁力を用いて構造物の鋼材に吸着することで,安定した点検を可能にした.このEPMと呼ばれる磁石の特徴は,磁力の吸着状態と開放状態を自在に制御することが可能なことである.また,この磁石は電磁石と異なり,吸着と離脱時のみ電力を消費し,吸着中は電力を消費しない.このUAVはEPMに加え,先端にカメラを有するアームも装備するため,EPMを用いて吸着した状態で安定した目視点検をすることが可能である.



\begin{figure}[H]
\begin{center}
\includegraphics[height=70mm]{./Fig2/Overall_picture_of_our_robot.png}
\caption{Overall view of the developed inspection robot in the previous study}
\label{}
\end{center}
\end{figure}

また,自身の先行研究ではこのEPMを吸着機構に採用した振動計測モジュールを開発した[].このモジュールは加速度センサ及びデータロガーを搭載し,対象物に吸着した後,加速度を計測,ロギングすることが可能である.その振動を解析することで対象物の健全性を評価することが可能である.点検方法としては,点検箇所までUAVを用いて運搬し,EPMの磁力で貼り付け,必要なデータを計測した後,再びUAVで回収する流れである.また,重量が約230 gであるため,赤堀らが開発したUAVなどであれば,複数個積載し,運搬することが可能である.


\begin{figure}[H]
  \begin{center}
    \begin{tabular}{c}

      \begin{minipage}{0.5\hsize}
        \begin{center}
         \includegraphics[height=50mm]{./Fig2/front.png}
	  \caption{Front view of vibration measurement unit}
        \end{center}
      \end{minipage}

      \begin{minipage}{0.5\hsize}
        \begin{center}
          \includegraphics[height=50mm]{./Fig2/front.png}
          \caption{Front view of vibration measurement unit}
        \end{center}
      \end{minipage}

    \end{tabular}
  \end{center}
\end{figure}

これらのUAVを用いた点検では,構造物の地形に対して柔軟な対応性を有するが,機体の操作は全て自身の手で行う必要性がある.経験の豊富なパイロットであれば,狙い通りの点検箇所に接近することやセンサを取り付けることが可能であるが,経験の浅いパイロットであると,そういったことは非常に困難である.

\subsection{研究目的}
上記の理由より,飛行型ロボットの有用性は高いが,操作面においてまだまだ課題が多いのが現状である.そこで,現在市販される多くのドローンはGPSなどの位置計測システムを用いて全自動若しくは半自動飛行が可能である[][][].本研究でも用いたフライトコントローラもGPSアンテナが搭載されており,屋外のGPS環境であれば,目的地を入力するだけでその地点まで全自動飛行することが可能である.しかし,研究対象とする橋梁などの構造物の周辺環境においては,遮蔽物によりGPSの電波が遮断されるため,GPSを用いた自動飛行は不可能である.そこで,非GPS環境においても自動飛行可能なUAVも開発が進んでいる.株式会社自律制御システム研究所はSLAM(Simultaneous Localization and Mapping)の機能を搭載したUAVを開発しており,非GPS環境においても自己位置推定と環境地図の作製を同時に行いながら自律飛行をすることが可能である.
	

\begin{figure}[H]
\begin{center}
\includegraphics[height=70mm]{./Fig2/photo5.png}
\caption{ACSL-PF1}
\label{photo5}
\end{center}
\end{figure}

手法としては搭載された下向きのモノラルカメラや前方に向けて取り付けられたステレオカメラを用いて撮影した画像に処理を施すことで自己位置や方角を推定し自律飛行を行う.しかし,それらをするためには高価で精度の高いカメラを複数個積むことや画像処理用のGPUを積む必要があるため,機体自体の値段が高価になる.また,画像から得た情報を基にあらゆる値を計算するには計算負荷の高い演算をしなくてはならない.点検に掛けられる財源が限られている自治体にとって,人件費がかからなかったとしても,機体が高価であった場合,大きなコスト削減は見込めない.また,多くのセンサや負荷が高い計算をするに伴って性能の高いCPUやGPUを搭載する必要があるため,機体が大型になってしまい,狭い点検空間に入っていくのが難しくなるといった課題がある.

そこで,それらの課題を克服するため,本研究ではローカル座標を推定する手法にUWBモジュールを用いる手法に着目した.UWB(超高帯域無線通信,Ultra Wide Band)とは無線通信方式の一種であり,Wi-FiやBlutoothなど他の無線通信方式に比べ広い帯域幅を持った通信方式である.これを利用し,電波を送信し返ってきた時間より距離を測定することが可能なため,UAVの位置を3次元空間内で推定することが可能である.模式図をFig. \ref{UWB_positioning}に示す.


\begin{figure}[H]
\begin{center}
\includegraphics[height=70mm]{./Fig2/UWB_positioning.png}
\caption{Schematic diagram of the position estimation method}
\label{UWB_positioning}
\end{center}
\end{figure}

詳細は以降の章にて後述するが,座標が既知の各固定局と移動局との距離を計測し,マルチラテレーションの手法を用いることで,移動局の座標を得ることが可能である.この手法はGPSアンテナとGPS衛星との距離を用いて移動局の座標を得るのと同手法である.またUWBモジュールは一個あたり数千円と非常に安価であり,機体のコスト削減に有効である.それに加えてマルチラテレーションの計算は画像解析を用いる計算などに比べて非常に負荷が軽く,高性能なコンピュータを乗せる必要がない.しかし,UWBモジュールの通信周期は4つの固定局を用いた場合に10 Hzと低く,クアッドロータが高速度で移動した際には位置推定周期としては不十分である.そこで,本研究での目的はUWBモジュールの更新周期の遅さを補うためにUWBモジュールを用いたローカルポジショニングシステムにIMU,距離センサ,及びオプティカルフローセンサの値も参照することでクアッドロータの位置を推定し,その位置を制御する手法を提案する.IMU(Inertial Measurement Unit)は加速度,角速度,地磁気を数1000 Hz程度の高い更新周波数で計測可能であり,加速度や角速度を積分することで速度や変位を得ることが可能である.しかし,値を積分した際に誤差が蓄積してしまう.また距離センサは計算することなく,直接地面との距離を計測することが可能である.最後にオプティカルフローセンサは地面の特徴点の移動量より機体の速度を算出することが可能である.先行研究において赤堀ら[]はUWBモジュール,IMU,距離センサを用いたクアッドロータの位置推定システム及び位置制御システムを開発した.しかし,姿勢角制御の周期の遅さが原因でクアッドロータの位置を制御することはなし得ていなかった.そこで,本研究では姿勢角制御の遅さを補うためにフライトコントローラを追加で導入し,改善を図った.次章からはUWBモジュール及び各種センサをカルマンフィルタに適用したクアッドロータの位置推定方法および位置制御方法を示していく.


\newpage

%%%%%%%%%%%%%%%%%%%%%%%%%%%%%%%%%%%%%%%%%%%%%%%%%%%%%%%%%%%%%%%%%%%%%%%%%%%%%%%%%%%%%%%%%%
%%%%%%%%%%%%%%%%%%%%%%%%%%%%%%%クアッドロータの位置推定アルゴリズム%%%%%%%%%%%%%%%%%%%%%%%%%%%%%%%%%
%%%%%%%%%%%%%%%%%%%%%%%%%%%%%%%%%%%%%%%%%%%%%%%%%%%%%%%%%%%%%%%%%%%%%%%%%%%%%%%%%%%%%%%%%%
\section{クアッドロータの位置推定アルゴリズム}
本研究で提案するクアッドロータの位置及び姿勢推定にはカルマンフィルタ,その中でも特に非線形なシステムに適応可能な拡張カルマンフィルタ(EKF : Extended Kalman Filter)を使用している.ここで,カルマンフィルタとは,誤差を含むある観測値(2種類以上も可)を使用し,時々刻々と変化する状態量を推定するのに用いられる,逐次処理が可能なフィルタリング手法である.代表例としてはカーナビゲーションシステムが挙げられる.カーナビゲーションシステムはGPSから得られる誤差の乗った位置情報と内部の加速度センサから得られる誤差の乗った加速度情報を統合することにより,自車の位置座標や速度などを推定している.本研究ではIMU,UWBモジュール,距離センサ,オプティカルフローセンサから得られるセンサ値を統合し,機体の姿勢角,速度,座標を推定するのに用いられる.本章では初めに拡張カルマンフィルタの原理を示した後,UWBモジュールやオプティカルフローセンサの原理やキャリブレーション方法について述べる.そして最後にそれらセンサを統合したシステムを示し,位置の推定方法を説明する.



\subsection{拡張カルマンフィルタの原理}
拡張カルマンフィルタは非線形システムを書く自国において線形化し,それぞれの時刻において時変カルマンフィルタを適用するという考えに基づいている.まず離散時間非線形状態空間表現が次の二式で表される場合を考える.

\begin{align}
\textbf{x}(t+1) = \textbf{f}(\textbf{x}(t)) + \textbf{B}\textbf{v}(t)
\label{2-1}
\end{align}

\begin{align}
\textbf{y}(t) = \textbf{h}(\textbf{x}(t)) + \textbf{w}(t)
\label{2-2}
\end{align}

\noindent
ここで$\textbf{f}(\cdot)$,および$\textbf{h}(\cdot)$はベクトル値をとる$\textbf{x}(t)$の非線形関数である.また$\textbf{v}(t)$は平均値ベクトル0,共分散行列$\textbf{Q}$のr次元システム雑音ベクトル,$\textbf{w}(t)$は平均値ベクトル0,共分散行列$\textbf{R}$のp次元観測雑音ベクトルであり,互いに独立な正規性白色雑音と仮定する.時刻$t$,$t+1$において,それぞれ事前状態推定値$\hat{\textbf{x}}^{-}$と事後推定値$\hat{\textbf{x}}$が利用可能であるという仮定のもとで,式\ref{2-1}と式\ref{2-2}の非線形関数をテイラー級数展開を用いて線形近似すると,

\begin{align}
\textbf{f}(\textbf{x}(t))=\textbf{f}(\hat{\textbf{x}}(t))+\textbf{A}(t)(\textbf{x}(t)-\hat{\textbf{x}}(t)),\left.\textbf{A}(t)=\frac{\partial \textbf{f}(\textbf{x})}{\partial \textbf{x}}\right|_{\textbf{x}=\hat{\textbf{x}}(t)}
\label{2-3}
\end{align}

\begin{align}
\textbf{h}(\textbf{x}(t))=\textbf{h}(\hat{\textbf{x}}^{-}(t))+\textbf{c}^{T}(t)(\textbf{x}(t)-\hat{\textbf{x}}^{-}(t)),\left.\textbf{C}^{T}(t)=\frac{\partial \textbf{h}(\textbf{x})}{\partial \textbf{x}}\right|_{\textbf{x}=\hat{\textbf{x}}^{-}(t)}
\label{2-4}
\end{align}\

\noindent
が得られる.式\eqref{2-3}を式\eqref{2-1}に,式\eqref{2-4}を式\eqref{2-2}に代入すると,それぞれ,

\begin{align}
\textbf{x}(t+1)=\textbf{A}(t)\textbf{x}(t)+\textbf{b}\textbf{v}(t)+\textbf{f}(\hat{\textbf{x}}(t))-\textbf{A}(t)\hat{\textbf{x}}(t)
\label{2-5}
\end{align}

\begin{align}
\textbf{y}(t)=\textbf{C}^{T}(t)\textbf{x}(t)+\textbf{w}(t)+\textbf{h}(\hat{\textbf{x}}^{-}(t))-\textbf{C}^{T}(t)\hat{\textbf{x}}^{-}(t)
\label{2-6}
\end{align}\

\noindent
が得られる.さらに

\begin{align}
\textbf{u}(t)=\textbf{f}(\hat{\textbf{x}}(t))-\textbf{A}(t)\hat{\textbf{x}}(t)
\label{2-7}
\end{align}

\begin{align}
\textbf{z}(t)=\textbf{y}(t)-\textbf{h}(\hat{\textbf{x}}^{-}(t))+\textbf{C}^{T}(t)\hat{\textbf{x}}^{-}(t)
\label{2-8}
\end{align}\

\noindent
とおくと,式\eqref{2-5}と式\eqref{2-6}はそれぞれ次のようになる.

\begin{align}
\textbf{x}(t+1)=\textbf{A}\textbf{x}(t)+\textbf{B}\textbf{v}(t)+\textbf{u}(t)
\label{2-9}
\end{align}

\begin{align}
\textbf{z}(t)=\textbf{C}\textbf{x}(t)+\textbf{w}(t)
\label{2-10}
\end{align}\

\noindent
このように非線形システムをテイラー級数展開を用いて線形近似すると制御入力がある場合のカルマンフィルタになる.以上より予測ステップにおける更新式は

\begin{align}
\hat{\textbf{x}}^{-}(t)=\textbf{f}(\hat{\textbf{x}}(t-1))
\label{2-11}
\end{align}

\begin{align}
\textbf{P}^{-}(t)=\textbf{A}(k-1)\textbf{P}(t-1)\textbf{A}^{T}(k-1)+\textbf{B}\textbf{Q}\textbf{B}^{T}
\label{2-12}
\end{align}\

\noindent
となる.ここで$\textbf{A}(t-1)$は

\begin{align}
\left.\textbf{A}(t-1)=\frac{\partial \textbf{f}(\textbf{x})}{\partial \textbf{x}}\right|_{\textbf{x}=\hat{\textbf{x}}(t)}
\label{2-13}
\end{align}\

\noindent
となる.またフィルタリングステップの更新式は

\begin{align}
\hat{\textbf{x}}(t)=\hat{\textbf{x}}^{-}(t)+\textbf{g}(t)(\textbf{y}(t)-\textbf{h}(\hat{\textbf{x}}^{-}(t)))
\end{align}

\begin{align}
\textbf{P}(t)=(\textbf{I}-\textbf{G}(k)\textbf{C}^{T}(t))\textbf{P}^{-}(k)
\end{align}\

\noindent
となる.ここでカルマンゲイン$G(t)$及び$\textbf{C}^{T}(t)$は次式で計算することができる.

\begin{align}
\textbf{G}(t)=\textbf{P}^{-}(k)\textbf{C}^{T}(\textbf{C}\textbf{P}^{-}(k)\textbf{C}^{T}+\textbf{R})^{-1}
\end{align}\

\begin{align}
\left.\textbf{C}^{T}(t)=\frac{\partial \textbf{h}(\textbf{x})}{\partial \textbf{x}}\right|_{\textbf{x}=\hat{\textbf{x}}^{-}(t)}
\end{align}\

\noindent
拡張カルマンフィルタにおいては各時刻ごとに$\textbf{f}$と$\textbf{h}$のヤコビアンを求めるため偏微分の計算を行わなければならない.そのため,微分可能な滑らかな非線形性の場合には拡張カルマンフィルタを適用できるが,不連続な非線形性をもつ場合には適用できない.






\subsection{UWB通信}
\subsubsection{UWB通信の測距原理(発散の回避方法,取付角の工夫)}
まず,無線通信を利用した距離測定の技術はBlutoothやWi-Fiを用いたものなどが挙げられる.片方のデバイスが電波を発し,それをもう片方が受け取り,デバイス同士が通信を行う.測距の方法は,受信側の受け取った電波の強弱により発信側が近くにあるのか遠くにあるのかを判別し,距離を算出する.しかし,弱い電波を受け取った際,距離が離れているため電波が弱いのか,それともデバイス間に障害物があるため電波が弱くなっているのか,といったことを判別できない.したがって,Wi-FiやBlutoothを用いた測距は誤差が5 mほど乗るとされている.一部の企業はWi-Fi電波の飛行時間(Time of flight,ToF)や到着時間(Time of Arrival,ToA)を使用し,距離をより正確に測るための代替アルゴリズムを開発したが,既存のWi-Fiを用いてこれを行うことは,ハードウェアの問題により難しいとされている.

UWB通信(超広帯域無線通信,Ultra Wide Band)とは無線通信方式の一種であり,非常に広い帯域幅にわたって電波を拡散させることで高速通信を可能とする.Fig. \ref{}にUWB通信と既存の無線通信方式とのスペクトラムの比較を示す.周波数の帯域は3.1 GHzから10.6 GHzであり,その帯域幅は7.5 GHzと従来の無線の数百 kHzやWi-Fiの20 MHzに比べてかなり広い帯域を利用している.

\begin{figure}[H]
\begin{center}
\includegraphics[height=60mm]{./Fig2/UWB_power.png}
\caption{Comparison of power spectrum between UWB and other wireless communication system}
\label{UWB_power}
\end{center}
\end{figure}

\noindent
送信出力は-41.3 dBm/MHz以下に制限されており,これは家庭用テレビやパソコン等の一般の電子機器等が発生する雑音レベルの約500分の1以下という非常に小さな出力である.また,UWB通信はFig. \ref{narrow_wide}に示すようにナノ秒オーダーのパルスを通信に使用し,狭帯域信号のような立ち上がりが遅い信号に比べて,デバイス間の距離を高精度で測定可能である.

\begin{figure}[H]
\begin{center}
\includegraphics[height=60mm]{./Fig2/narrow_wide.png}
\caption{Comparison of pulses between narrow band and ultra wide band signals}
\label{narrow_wide}
\end{center}
\end{figure}

\noindent
また,Fig. \ref{narrow},Fig. \ref{wide}に示すように狭帯域信号に比べ,UWB信号はSN比が良いため,ノイズの影響を受けにくいことなどが特長として挙げられる.

\begin{figure}[H]
  \begin{center}
    \begin{tabular}{c}

      \begin{minipage}{0.5\hsize}
        \begin{center}
         \includegraphics[height=50mm]{./Fig2/narrow.png}
	  \caption{Narrow band signal with noise}
	 \label{narrow}
        \end{center}
      \end{minipage}

      \begin{minipage}{0.5\hsize}
        \begin{center}
          \includegraphics[height=50mm]{./Fig2/wide.png}
          \caption{Ultra wide band signal with noise}
	   \label{wide}
        \end{center}
      \end{minipage}

    \end{tabular}
  \end{center}
\end{figure}

\noindent
無線通信を利用する際に大きな問題とされるのがマルチパスの発生である.マルチパスとは電波を受信した際に,直接波の他に壁などの障害物に反射した反射波を受信してしまう場合があり,この際,正確な信号受け取れないといったことや,直接波の位相を反射波が遅らせてしまうといったことが発生する.テレビやラジオにてゴースト障害が起きるのも,このマルチパスが原因とされている.無線通信を利用した測距の場合には,正確な到達時間を計測できず,測距誤差が発生するということが起きる.次のFig. \ref{narrow_with_reflections},Fig. \ref{wide_with_reflections}に狭帯域信号とUWB信号のマルチパスの影響の比較を示す.



\begin{figure}[H]
  \begin{center}
    \begin{tabular}{c}

      \begin{minipage}{0.5\hsize}
        \begin{center}
         \includegraphics[height=50mm]{./Fig2/narrow_with_reflections.png}
	  \caption{Narrow band signal with reflections}
	 \label{narrow_with_reflections}
        \end{center}
      \end{minipage}

      \begin{minipage}{0.5\hsize}
        \begin{center}
          \includegraphics[height=50mm]{./Fig2/wide_with_reflections.png}
          \caption{Ultra wide band signal with reflections}
	   \label{wide_with_reflections}
        \end{center}
      \end{minipage}

    \end{tabular}
  \end{center}
\end{figure}

\noindent
狭帯域信号では立ち上がりと立ち下がりが鈍く,障害物で位相が反転した反射波の影響により直接波の位相や振幅が変化させられているのが分かる.しかし,UWB信号は立ち上がり時間と立ち下がり時間が短いため,直接波の後に反射波が来ていても直接波に悪影響を与えない.このようなマルチパス耐性が高いといったこともUWBの特長である.


次にUWB通信を用いた測距方法について見ていく.UWB通信による測距は主に電波の飛行時間(ToF)を用いている.送信側と受信側が互いに通信し,電波の到達するのに掛かった時間(Tof)に電波の速度($c$)を掛けることにより,距離($d$)を測定することが可能である.次に計算式を示す.

\begin{align}
d = \textrm{ToF} \times c
\label{2-2-1}
\end{align}

\noindent
またFig. \ref{UWB_ranging_2}に通信プロトコルを示す.

\begin{figure}[H]
\begin{center}
\includegraphics[height=60mm]{./Fig2/UWB_ranging_2.png}
\caption{Ranging protocol of UWB module}
\label{UWB_ranging_2}
\end{center}
\end{figure}

\noindent
これ以降,クアッドロータに搭載された移動局をタグ,床に固定された固定局をアンカと記載する.クアッドロータの座標を3次元で推定するためには1つのタグと3つ以上のアンカが必要である.それに伴い,複数のデバイスを識別するためにまずアンカ側から16進数からなるIDを送信する.その後タグ側から応答があり,最後にアンカ側のタイムスタンプ($T_{SP}, T_{RR}, T_{SF}$)を送信する.この送られてきたタイムスタンプ及びタグ側のタイムスタンプ($T_{RP}, T_{SR}, T_{RF}$)を用い,次式のように電波の飛行時間(ToF)を計算可能である.

\begin{align}
\textrm{ToF} = \frac{1}{4}((T_{SP} - T_{RR}) - (T_{SR} - T_{RP}) + (T_{RF} - T_{SR}) - (T_{SF} - T_{RR})) 
\label{2-2-2}
\end{align}


次のFig. \ref{Anchor}に製作したUWBモジュールのアンカを示す.搭載されたUWBチップはDecaWave社製のDWM1000である.マイコンはTeensy3.6を用いて制御を行っている.電源はFig. \ref{Li_ion}に示すUltraLife社製, 1 cell, 1800 mAhのリチウムイオンバッテリを用いている.図の基板構成で満充電から約3時間容量が保つ.Fig. \ref{Anchor}からわかるように,小型であるためアンカの設置や回収は容易に行うことが可能である.


\begin{figure}[H]
\begin{center}
\includegraphics[height=60mm]{./Fig2/Anchor.png}
\caption{UWB anchor}
\label{Anchor}
\end{center}
\end{figure}

\begin{figure}[H]
\begin{center}
\includegraphics[height=50mm]{./Fig2/Li_ion.jpg}
\caption{Battery for UWB Anchor}
\label{Li_ion}
\end{center}
\end{figure}



前述したようにUWB通信はマルチパスに耐性があり,また[]でも示されているように,壁のように強い反射体の近くにおいても測距性能は正確であるが,森林のようなデバイス間に障害物が存在する環境:NLOS(Non Line Of Sight,非見通し環境)においてはマルチパスが発生し,真値より大きく外れた値や負の値を取ることがある.本研究においても,このマルチパスの影響を確認した.Fig. \ref{UWB_out}に測距した際にマルチパスが発生し,値が発散した結果を示す.またそれの拡大図もFig. \ref{UWB_out_zoom}に合わせて示す.

\begin{figure}[H]
\begin{center}
\includegraphics[height=60mm]{./Fig2/UWB_out.png}
\caption{Ranging result when outliers occur}
\label{UWB_out}
\end{center}
\end{figure}

\begin{figure}[H]
\begin{center}
\includegraphics[height=60mm]{./Fig2/UWB_out_zoom.png}
\caption{Ranging result when outliers occur (enlarged image above)}
\label{UWB_out_zoom}
\end{center}
\end{figure}

\noindent
測距手順はまず,実験室の床に距離を測り,真値の点を1 mから5 mまで1 m間隔で取った.そして,その点に合わせる形で4つのアンカを1 mずつ動かし,それぞれのアンカと固定したタグとの間の測距をUWBモジュールにより計測した.2つの図からわかるように,測距した最長の距離は5 mであるのにも関わらず,値が発散し最長300 mほどの距離や負の距離を計測している.UWBモジュール間に障害物が無かったのにも関わらず,こういったマルチパスが発生した原因は,壁や天井により電波が反射したためであると考えられる.タグと各アンカとの正確な距離が得られていない場合,3次元空間内におけるクアッドロータの位置推定値も不正確になるため,こういった発散した値は取り除く必要がある.そこで,まず初めに行ったのが電波の指向性を考慮したモジュールの改良である.UWBモジュールには指向性があり,電波強度の強い向きと弱い向きが$x,, y, z$軸の回転方向に存在する.本研究にて採用したUWBチップ(DWM1000)の軸を次のFig. \ref{UWB_axis}のように定義する.


\begin{figure}[H]
\begin{center}
\includegraphics[height=40mm]{./Fig2/UWB_axis.png}
\caption{UWB chip axis}
\label{UWB_axis}
\end{center}
\end{figure}

\noindent
DWM1000を製造するDecaWave社がチップから発せられる電波の指向性をデータシートにおいて公開している.$x, y, z$軸周りにおける電波の指向性をFig. \ref{UWB_angle_x}からFig. \ref{UWB_angle_z}に示す.

\begin{figure}[H]
\begin{center}
\includegraphics[height=50mm]{./Fig2/UWB_angle_x.png}
\caption{Directivity of radio wave around $x$ axis}
\label{UWB_angle_x}
\end{center}
\end{figure}

\begin{figure}[H]
\begin{center}
\includegraphics[height=50mm]{./Fig2/UWB_angle_y.png}
\caption{Directivity of radio wave around $y$ axis}
\label{UWB_angle_y}
\end{center}
\end{figure}

\begin{figure}[H]
\begin{center}
\includegraphics[height=50mm]{./Fig2/UWB_angle_z.png}
\caption{Directivity of radio wave around $z$ axis}
\label{UWB_angle_z}
\end{center}
\end{figure}

\noindent
3つの図から考慮すると,UWBチップの平面方向から垂直な方向,つまりFig. \ref{UWB_axis}の$x$軸の方向が通信対象を向けば,電波強度を確保できることがわかる.そこで,Fig. \ref{UWB_case_angle}に示すアンカ用のホルダを3Dプリンタにより製作した.


\begin{figure}[H]
\begin{center}
\includegraphics[height=50mm]{./Fig2/UWB_case_angle.png}
\caption{UWB module holder considering radio wave directivity}
\label{UWB_case_angle}
\end{center}
\end{figure}

\noindent
これを用いる前は,実験室の壁面に対して垂直にアンカを貼り付けていたが,このホルダを用いることでチップ垂直方向をクアッドロータの方向付近を向けることが可能である.これを垂直な壁面に貼り付けることで以前より$30^\circ$ほどオフセット角を稼ぐことができる.そしてこのホルダをアンカの4つ分使用し,実験室の4隅に取り付け,アンカで取り囲まれる範囲内でタグを動かし,測距を行った.アンカの電波の強い方向がタグの方を向くため,マルチパスは発生しなくなると考えられたが,結果はFig. \ref{UWB_out}と同様に発散した値を計測した.したがってアンカの取り付け方法によってマルチパスの発生を抑えることは難しいと考えた.

次に行ったのがプログラムによる発散値の処理を行い,発散値をロギングしないという方法である.処理の内容は,アンカとタグとで通信を行い,計測した距離データを一度メモリ内にバッファする.そして,次の距離データが送られてきた際にバッファした距離と比較し,閾値を上回る変化があった場合にはその直近のデータを破棄し,前のデータを採用するというものである.距離データの発散は,条件にもよるが数秒に1サンプル程度の頻度で発生し,連続したサンプル同士が両方発散することはない.したがって,この処理の方法により真値からかけ離れた値のみを適切に無視することが可能である.閾値としては,クアッドロータが$1 m/s$の速度で水平面内を飛行可能であると考慮し,測距の1ループ内で移動不可能な距離を閾値として設定した.このプログラムによる処理を行い,発散値を回避した測距結果をFig. \ref{UWB_cali}に示す.

\begin{figure}[H]
\begin{center}
\includegraphics[height=60mm]{./Fig2/UWB_cali.png}
\caption{Ranging result avoiding outlier}
\label{UWB_cali}
\end{center}
\end{figure}

\noindent
実験方法はFig. \ref{UWB_out}の実験を行った際と同様である.アンカを動かした際の振動による値のずれは生じているが,処理する前の結果と比べて適切に発散値を回避することができている.


\subsubsection{更新周波数の改善のアプローチ(マイコンスペックの比較,周波数グラフの比較)}

\subsection{オプティカルフローセンサ}
\subsubsection{オプティカルフローセンサの原理(センサ座標系のワールド座標系への変換)}
先行研究では,クアッドロータの位置や速度などの状態量をUWBモジュール及びIMUを用いて推定していた.しかし,UWBモジュールのアンカを4つ用いた際に得られる距離データの更新周期は10 Hzと遅く,クアッドロータが高速度で動いた際や運動の不安定さが原因の細かな振動が生じた際に更新周期が不十分であった.そこで,更新周期の遅さを補うために本研究からは新たにオプティカルフローセンサを搭載した位置推定および位置制御システムを構築した.オプティカルフローセンサとは,カメラで撮影した画像の特徴点の移動量から,移動する物体の速度や移動量を算出可能なセンサである.オプティカルフローはPython環境にてOpenCVなどのモジュールを用い,パーソナルコンピュータを使用し計算することも多い.例としては,あらかじめ撮影した車の写真から車速を割り出すことなどに使用される.そして,オプティカルフローセンサは画像の撮影から移動量の算出までを,搭載したチップで行うことが可能である.そして,それらのセンサ値をマイコンにより適切に処理することにより,搭載したロボットなどの速度や移動量を得ることができる.また,別の例としては光学式マウスはこのセンサにより移動量を算出している.[]や[]においてもクアッドロータにこのセンサを搭載し,クアッドロータの位置保持制御や目標点まで移動させる制御などを行っている.本研究で採用したオプティカルフローセンサモジュールの外観をFig. \ref{optical_flow}に示す.


\begin{figure}[H]
\begin{center}
\includegraphics[height=50mm]{./Fig2/optical_flow_with_axis.png}
\caption{Optical Flow Sensor Module}
\label{optical_flow}
\end{center}
\end{figure}

\noindent
このモジュールはオプティカルフローセンサとしてPMW3901のチップを搭載しており,それに加え距離センサとしてVL53L0x ToFセンサを搭載している.二つのセンサが同一の基板上に搭載されているため汎用性が高いモジュールであり,オプティカルフローセンサはSPI通信,距離センサは$\textrm{I}^{2}\textrm{C}$通信をそれぞれ用いるため,配線が少なくてすむ.また,図より大きさもとても小型であることがわかり,積載量の限られたクアッドロータなどにおいても容易に搭載可能である.後述するが,距離センサはオプティカルフローセンサから得た値を補正するのに用いられる.


\subsubsection{センサのキャリブレーション}
オプティカルフローセンサから得られる速度のキャリブレーション方法について述べる.まず,オプティカルフローセンサから得られる値は撮影した写真の変化から得られる特徴点の移動量,つまり単位は距離ではなくピクセルである.例えば,ある物体の上でクアッドロータが100 mm移動したとする.その際,クアッドロータの高さが低ければピクセル数の変化量は大きいが,高さが二倍になった場合,ピクセル数の変化量は二分の一になる.それを表した図がFig. \ref{opt_distance}である.

\begin{figure}[H]
\begin{center}
\includegraphics[height=70mm]{./Fig2/opt_distance.png}
\caption{Difference in Sensor Value Due to Difference in Quad-Rotor Height}
\label{opt_distance}
\end{center}
\end{figure}

\noindent
したがって,オプティカルフローセンサから得られるピクセル数の変化量と距離センサから得られる距離を乗算し,適切な定数を掛けることで真の移動距離を算出することができる.なお,飛行体の高度とオプティカルフローセンサから得られる値の関係は[]に詳しい記述がある.また[]を参考にすると,センサが値を取得し,次の値を取得するまでをセンサの1ループと考えると,その1ループ中に移動した距離は次式で表せる.

\begin{align}
\textrm{Moving distance} = \left( \frac{\textrm{Sensor value} \times \textrm{Altitude}}{\textrm{Sensor's resolution in pixels} \times \textrm{Scalar}}\right) \times 2.0 \times \tan \left(\frac{\textrm{Field of view}}{2.0} \right)
\label{2-3-1}
\end{align}

\noindent
Sensor valueはオプティカルフローセンサから得た値であり,Altitudeは距離センサから得た地面までの距離である.Field of viewはオプティカルフローセンサが撮影可能な画角であり,Sensor's resolution in pixelsは撮影画像の画素数つまり解像度である.この二つの値はセンサ固有のものであり,それらを次のTable. \ref{sensor_value}に示す.


\begin{table}[H]
 \caption{Sensor characteristic value}
  \label{sensor_value}
 \begin{center}
  \begin{tabular}{|c||c|}
   \hline
    FOV [$^{\circ}$]&  42 \\
   \hline
  Resolution [pixel]& 30 $\times$ 30       \\
   \hline 
  \end{tabular}
 \end{center}
\end{table}

\noindent
Scalarは定数であり,このScalarの値を次のキャリブレーションにより求めた.式(\ref{2-3-1})から得られるのは,あくまでも1ループ中に動いた距離,つまり速度である.物体が移動した真の速度を用いてこのScalarの値を求めることは可能であるが,速度を外部のセンサなどを用いて求めるのはシステムが複雑になる.そこで,速度ではなく真の移動距離を用いてこのScalarの値を求めることにした.過去のある時点から現在までに移動した距離を算出するには,センサから得られた速度に対してマイコンが1ループにかかる処理時間を掛け,それを足し合わせることで可能である.つまり,センサ値を積分することである.そこで次のような実験を行った.ある高さに対し,センサが下を向くように固定し,一方向のみレールにより自由度を与える.レールに既知の距離を複数点マークし,その距離分だけセンサを動かした.センサ値を積分して得られた距離データと真の距離を最小二乗法により近似することでScalarを求めた.センサを動かした距離はTable \ref{Sensor_Value_and_Moved_Distance}に示すように0 mmから800 mmまでの200 mm間隔である.固定したセンサの床からの高さは775 mmである.また,センサ値の値は三回試行を行った上でのそれらの平均値である.



\begin{table}[H]
 \caption{Sensor values and moving distance}
 \label{Sensor_Value_and_Moved_Distance}
 \begin{center}
  \begin{tabular}{|c|c|}
   \hline
    Moving distance [mm] & Average of sensor value [-]\\
   \hline\hline
   0 & 0 \\
   \hline 
   200 & 3061 \\
   \hline
   400 & 6165\\
   \hline
   600 & 9198 \\
   \hline
   800 & 12296 \\
   \hline
  \end{tabular}
 \end{center}
\end{table}

\noindent
そして,結果をFig. \ref{opt_cali}に示す.


\begin{figure}[H]
\begin{center}
\includegraphics[height=60mm]{./Fig2/opt_cali.png}
\caption{Relationship between sensor value and moving distance}
\label{opt_cali}
\end{center}
\end{figure}

\noindent
近似直線からわかるように,センサ値と移動距離の関係は線形的に近似可能である.センサの床からの距離,つまり式(\ref{2-3-1})中のAltitudeは距離センサ得られる値を用いたが,Altitudeの値を真値(775 mm)に固定して計算しても結果は同様のものであった.また,この結果はFig. \ref{optical_flow}に示す$y$軸方向の移動の実験におけるものであったが,同様に$x$軸方向における実験も行ったが結果は同様のものであった.したがって,各軸におけるScalarの値は同一の値を採用した.以上の検証よりScalarが求まったため,オプティカルフローセンサよりセンサ座標系における速度が得られるようになった.

次にセンサ座標系のワールド座標系への変換方法について述べる.前述した方法で得られるのはセンサ座標系における速度であり,クアッドロータの位置や速度をワールド座標系内で推定し,位置を制御するにはセンサ座標系をワールド座標系へ変換する必要がある.まず,Fig. \ref{quad_fig}にワールド座標系と機体座標系の関係を示す.


\begin{figure}[H]
\begin{center}
\includegraphics[height=60mm]{./Fig2/quad_fig.png}
\caption{Relationship between the world coordinate system and the quad-rotor coordinate system}
\label{quad_fig}
\end{center}
\end{figure}


\noindent
ワールド座標系は地上の任意の点を原点とする地上に固定されている座標系であり,r-frameと表す.そして,機体座標系は機体重心を原点とし,機体前方を$y_b$,右方向を$x_b$,上方向を$z_b$とした座標系であり,b-frameと表す.また,3次元の姿勢表現としてオイラー角を用いた表現を行う.r-frameとb-frameの回転角の関係は$\phi, \theta, \psi$の3つのオイラー角を用いて表現できる.$\phi, \theta, \psi$はそれぞれ$x, y, z$軸周りにおける回転であり,ピッチ角,ロール角,ヨー角と呼ばれる.もし,機体がワールド座標系内にて$z$軸周りに$\psi$だけ回転した場合,オプティカルフローセンサから得られる速度をワールド座標系へ変換しなければ,ワールド座標系内での速度が得られない.オプティカルフローセンサから得られる$x, y$軸それぞれの速度を$\textbf{v}_o = [v_{ox}, v_{oy}]^T$とし,変換した後のワールド座標系内での速度を$\textbf{v}_r = [v_{rx}, v_{ry}]^T$とすると,それらの関係は$z$軸周りにおける回転行列を用いて次式により表せる.

\begin{align}
\label{2-3-1}
\begin{split}
\left[
    \begin{array}{c}
      v_{rx}  \\
      v_{ry}
    \end{array}
\right]
&= 
\left[
    \begin{array}{cc}
      \cos\psi & -\sin\psi \\
      \sin \psi & \cos\psi
    \end{array}
\right]
\left[
	\begin{array}{cc}
		v_{ox}  \\
		v_{oy}
	\end{array}
\right] \\
&=
\left[
    \begin{array}{c}
      v_{ox}\cos\psi  -v_{oy}\sin\psi \\
      v_{ox}\sin\psi + v_{oy}\cos\psi
    \end{array}
\right]
\end{split}
\end{align}\

\noindent
$z$軸周りの回転角度は地磁気及び角速度センサを使用して推定したものを用いる.推定方法は次章にて説明する.式(\ref{2-3-1})を用いることでクアッドロータのワールド座標系内での速度をオプティカルフローセンサにより得られるようになった.


\subsection{UWBモジュール,IMU,距離センサのみを用いた位置推定システム}
まず,先行研究において開発され,本研究において改良したクアッドロータの外観をFig. \ref{quad-rotor}に示す.また詳細な仕様をTable \ref{Quad-rotor specifications}に示す.

\begin{figure}[H]
\begin{center}
\includegraphics[height=80mm]{./Fig2/quad-rotor.png}
\caption{Quad-rotor equipped with UWB module, IMU and distance sensor}
\label{quad-rotor}
\end{center}
\end{figure}


\begin{table}[H]
 \caption{Quad-rotor specifications.}
 \label{Quad-rotor specifications}
 \begin{center}
  \begin{tabular}{|c||c|}
   \hline
   Total weight [kg] & 0.65 \\ 
   \hline
   Height [mm]& 110 \\
   \hline
   Width [mm]& 250 \\
   \hline
   Depth [mm]& 250 \\
   \hline
   Propeller size [inch] & 5$\times$3\\
   \hline
   Maximum flight time [min]& 10 \\
   \hline
    KV value [rpm/V]& 2633 \\
   \hline
   Buttery capacity[mAh]& 1800 \\
   \hline
  \end{tabular}
 \end{center}
\end{table}

\noindent
先行研究からのクアッドロータの変更点は新たにフライトコントローラを追加した点である.次のFig. \ref{system}に先行研究にて開発された位置推定アルゴリズムを用いた位置制御用のシステムを示す.

\begin{figure}[H]
\begin{center}
\includegraphics[height=80mm]{./Fig2/system.png}
\caption{System configuration of sensors}
\label{system}
\end{center}
\end{figure}

\noindent
本クアッドロータでは姿勢角や位置などの推定値の計算や位置制御のための姿勢角の計算などは全てLinux OS搭載のコンピュータ,Raspberry Piを用いて行う.センサボードはNAVIO2を搭載しており,加速度センサ及び角速度センサの機能を有している.UWBモジュールはSPI通信でArduino Pro Miniと通信しており,4つのアンカから集めた距離データを集約しUART通信でRaspberry Piに送っている.距離センサの値は$\textrm{I}^2\textrm{C}$通信を用いてRaspberry Piに送っている.先行研究においては,推定した自己位置や姿勢を基に姿勢角を制御する際,フライトコントローラとしてRaspberry Piを用い,Integral Backstepping制御を適用することでクアッドロータの位置制御を行っていた.しかし,その制御方法では姿勢角制御のループ周波数が遅く,位置制御が振動的になり目標の位置にクアッドロータを収束させることが不可能であった.そこで,本研究ではその姿勢角制御の遅さを補うため,推定値の演算を行うコンピュータとクアッドロータの姿勢角制御を行うコンピュータを分離させた.推定値の演算や位置の偏差から目標姿勢角を計算することなどは従来通りRaspberry Piを用いて行うが,姿勢角制御はmRobotics社が製作しているPixracerと呼ばれるフライトコントローラを用いて行うように仕様を変更した.Fig. \ref{pixracer}にPixracerの外観を示す.

\begin{figure}[H]
\begin{center}
\includegraphics[height=80mm]{./Fig2/pixracer.png}
\caption{Appearance of Pixracer}
\label{pixracer}
\end{center}
\end{figure}

\noindent
このボードは6個のPWM出力ピンを備えており,重量も約11 gと軽量のため,小型の航空機や地上機などのコントローラとして採用されることが多い.加速度センサ,角速度センサ,地磁気センサ,気圧センサといった各種センサやWi-Fiモジュールも搭載されるため,汎用性が高いコントローラでもある.また,GPSも搭載されており,GPS環境下であればオートフライトも可能である.本研究にてこのコントローラを採用した理由は非常に小型・軽量であるため,従来のクアッドロータに追加で搭載した場合にも,大きな改良を施さなくて良い点やファームウェアやプログラムがオープンソースなため,自身で内部の仕様を変更可能であることなどが理由である.また,何よりも大きな利点は高速なCPUを搭載し,完成度の高い姿勢角制御のアルゴリズムが使用可能なため,自身で構築した姿勢角制御アルゴリズムよりも精度良くクアッドロータの姿勢角を制御可能なことである.Fig. \ref{system}に示すようにフライトコントローラはUSBを介してRaspberry Piと接続されており,Raspberry Piが計算した目標姿勢角をフライトコントローラに送ることで,クアッドロータを目標姿勢に追従させることが可能である.Fig. \ref{system_core_design}に位置制御のブロック線図を示す.

\begin{figure}[H]
\begin{center}
\includegraphics[height=100mm]{./Fig2/system_core_design.png}
\caption{Block diagran of positon control system}
\label{system_core_design}
\end{center}
\end{figure}

\noindent
手順としてはまず,UWBモジュールのタグと各アンカとの距離($d_1$から$d_4$),距離センサから得た地面との距離($d_h$),センサボードから得た加速度($a_x, a_y, a_z$),角速度値($\omega_x, \omega_y, \omega_z$)をミキシングし,拡張カルマンフィルタを適用することで位置($x, y, z$),速度($v_x, v_y, v_z$),姿勢角($\phi, \theta, \psi$)を推定する.そして,目標座標と現在座標の偏差をPID制御器に通すことで目標姿勢角($\phi_d, \theta_d, \psi_d$)を求める.そしてそれをフライトコントローラへ送信可能なサーボ信号(Ch1からCh4)へと変換し,フライトコントローラへ送ることでクアッドロータの姿勢角及びスラストが制御され,目標位置へと機体が移動するという流れである.

次に拡張カルマンフィルタを用いてクアッドロータの位置を推定する方法について述べる.まず,姿勢角や位置を推定するのに拡張カルマンフィルタを用いる理由について述べる.初めに姿勢角についてであるが,姿勢角はカルマンフィルタの状態方程式側にて角速度センサの値を積分して角度に変換する.しかし,積分した際,角速度センサに乗ったノイズまで積分するため,推定した角度が時間とともにドリフトしてしまう.また,観測方程式側では加速度センサから得られる値よりヨー角以外の姿勢角を直接計算できる.その際,積分を行わないためドリフトをすることはないが,クアッドロータが並進移動した際に余計な加速度がかかるため,正確な角度を計算できなくなってしまう.それら二種類のセンサの不利な点を補うのに今回のようなカルマンフィルタを用いた姿勢角の推定は有効である.次に位置推定についてであるが,位置推定の計算は状態方程式側において加速度センサの値を二回積分し,移動距離を求め,観測方程式側ではUWBモジュールを用いて求めた距離を用い,マルチラテレーションの手法にて位置を推定している.加速度センサからは数千Hzにて加速度を得られるため,位置推定値も変化に対する応答性が高いが,姿勢角を求めた際と同様に,積分した際にドリフトが生じてしまう.また,UWBを用いた推定では正確な絶対座標が得られるが,UWBアンカを4つ使用した際の更新周波数が10 Hzと遅いため,クアッドロータが高速で運動した際には応答性の悪い推定になってしまう.このように位置推定においても,お互いのセンサ不利な点を補えるためカルマンフィルタは有効である.UWBモジュールを用いてクアッドロータの位置を推定する方法にマルチラテレーションの手法を用いていると述べたが,これについて説明する.マルチラテレーションとは座標が既知の3つ以上の固定局と移動局間の距離を求め,アルゴリズムに基づいて計算することで,移動局の座標を求める手法のことである.身近な例では,複数個のGPS衛星とGPS移動局との距離よりモジュールの位置を割り出すのに使用されている.位置を計算する手法は数多く存在し,[]や[]に詳しい解法が載っている.Fig. \ref{UWB_positioning2}にマルチラテレーションにおける模式図を示す.

\begin{figure}[H]
\begin{center}
\includegraphics[height=80mm]{./Fig2/UWB_positioning2.png}
\caption{Position relationship between Anchors and Tag}
\label{UWB_positioning2}
\end{center}
\end{figure}

\noindent
アンカが固定局のことであり,その絶対座標は既知である.そして,タグは座標未知の移動局であり,クアッドロータに搭載されている.このタグの位置を求めることがマルチラテレーションのことである.タグ及びアンカの座標とモジュール間の距離の関係は次式により表せる.

\begin{align} 
\label{2-4-1}
\left[
    \begin{array}{c}
      d_{1} \\
      d_{2} \\
      d_{3} \\
      d_{4}
    \end{array}
\right]
=
\left[
    \begin{array}{c}
      \sqrt{(\hat{x}-x_{1})^{2}+(\hat{y}-y_{1})^{2}+(\hat{z}-z_{1})^{2}} \\
      \sqrt{(\hat{x}-x_{2})^{2}+(\hat{y}-y_{2})^{2}+(\hat{z}-z_{2})^{2}} \\
      \sqrt{(\hat{x}-x_{3})^{2}+(\hat{y}-y_{3})^{2}+(\hat{z}-z_{3})^{2}} \\
      \sqrt{(\hat{x}-x_{4})^{2}+(\hat{y}-y_{4})^{2}+(\hat{z}-z_{4})^{2}} 
    \end{array}
\right]
\end{align}\



\noindent
方程式(\ref{2-4-1})を$\hat{x}, \hat{y}, \hat{z}$について解くことによりタグの三次元位置座標を求めることが可能である.なお,この式は後に出てくる拡張カルマンフィルタの観測方程式の一部である.次に拡張カルマンフィルタにおける状態方程式を見ていく.機体の速度ベクトル$\textbf{v}_\textbf{r} = [v_x, v_y, v_z]^T$と機体の位置ベクトル$\textbf{p}_\textbf{r} = [x_r, y_y, z_r]^T$ の関係は次式で表せる.

\begin{align}
\label{2-4-2}
\frac{d}{dt}\textbf{p}_\textbf{r}=\textbf{v}_\textbf{r}
\end{align}

\noindent
速度ベクトルの時間微分$\frac{d}{dt} \bf{v_r}$は加速度センサより得られるセンサ値$\textbf{a}_\textbf{b} = [a_x, a_y, a_z]^T$と回転行列$\textbf{R}$を用いて次式で表せる.


\begin{align}
\label{2-4-3}
\frac{d}{dt}\textbf{v}_\textbf{r} = \textbf{R}\textbf{a}_\textbf{b} - \textbf{g}_\textbf{r}
\end{align}

\noindent
ここで,ワールド座標系において重力は$z$軸負方向に重力加速度: $g$分だけ作用するため,$\textbf{g}_\textbf{r}$は$\textbf{g}_\textbf{r} = [0, 0, -g]^T$と定義できる.また,Fig. \ref{quad_fig}におけるr-frameとb-frame間の回転行列$\textbf{R}_\textbf{r}^\textbf{b}$は次式で定義される.

\begin{align}
\label{2-4-4}
\begin{split}
\textbf{R}_\textbf{r}^\textbf{b}(\phi,\theta,\psi)
&=
\left[
    \begin{array}{ccc}
      1 & 0 & 0\\
      0 & \cos\phi & \sin\phi \\
      0 & -\sin\phi & \cos\phi
    \end{array}
\right]
\left[
    \begin{array}{ccc}
      \cos\theta & 0 & -\sin\theta\\
      0 & 1 & 0 \\
      \sin\theta & 0 & \cos\theta
    \end{array}
\right]
\left[
    \begin{array}{ccc}
      \cos\psi & \sin\psi & 0\\
      -\sin\psi & \cos\psi & 0 \\
      0 & 0 & 1
    \end{array}
\right]
\\
&=
\left[
    \begin{array}{ccc}
      \cos\theta\cos\psi & \cos\theta\sin\psi & -\sin\theta\\
      \sin\phi\sin\theta\cos\psi & \sin\phi\sin\theta\sin\psi+\cos\phi\cos\psi & \sin\phi\cos\theta \\
      \cos\phi\sin\theta\cos\psi & \cos\phi\sin\theta\sin\psi-\sin\phi\cos\psi & \cos\phi\cos\theta
    \end{array}
\right]
\end{split}
\end{align}

\noindent
そしてオイラー角$\Theta = [\phi, \theta, \psi]^T$の時間微分と角速度センサより得られる機体角速度$\boldmath{\omega} = [p, q, r]^T$の関係は次式で表せる.


\begin{align}
\label{2-4-5}
\frac{d}{dt}\Theta = \textbf{W} \omega
\end{align}\

\noindent
なお,$\textbf{W}$は[]より

\begin{align}
\label{2-4-6}
\textbf{W}=
\left[
    \begin{array}{ccc}
      1 & \sin\phi\tan\theta & \cos\phi\tan\theta\\
      0 & \cos\phi & -\sin\phi \\
      0 & \sin\phi /\cos\theta & \cos\phi / \cos\theta
    \end{array}
\right]
\end{align}\

\noindent
である.状態ベクトルを$\textbf{x} = [\Theta, \textbf{p}_\textbf{r}, \textbf{v}_\textbf{r}]^T$と定義し,式(\ref{2-4-2})(\ref{2-4-3})(\ref{2-4-5})をまとめると次のようになる.

\begin{align}
\label{2-4-7}
\dot{\textbf{x}}(t)=\textbf{f}(\textbf{x}(t))
\end{align}\

\noindent
ただし$\textbf{f}(\cdot)$は


\begin{align} 
\label{2-4-8}
\textbf{f}(\textbf{x}(t))
=
\left[
    \begin{array}{c}
      \textbf{W}\omega \\
      \textbf{v}_\textbf{r} \\
      \textbf{R}\textbf{a}_\textbf{b} - \textbf{g}_\textbf{r} 
    \end{array}
\right]
\end{align}

\noindent
角速度センサ及び加速度センサから得られる出力値に含まれる観測ノイズ$\delta\omega=[\delta p,\delta q, \delta r]^{T}$及び$\delta \textbf{a}_\textbf{b}=[\delta a_{x},\delta a_{y}, \delta a_{z}]^{T}$を考慮すると式(\ref{2-4-7})は次式のようになる.

\begin{align}
\label{2-4-9}
\dot{\textbf{x}}(t)=\textbf{f}(\textbf{x}(t))+\textbf{B}\textbf{v}
\end{align}

\noindent
ただし,$\textbf{B},\textbf{v}$はそれぞれ


\begin{align} 
\label{2-4-10}
\textbf{B}
=
\left[
    \begin{array}{ccc}
      \textbf{0}_{3\times 3}&\textbf{0}_{3\times 3}&\textbf{I}_{3\times 3} \\
      \textbf{I}_{3\times 3}&\textbf{0}_{3\times 3}&\textbf{I}_{3\times 3}
    \end{array}
\right]^{T},\quad
\textbf{v}=[\delta p,\delta q, \delta r,\delta a_{x},\delta a_{y}, \delta a_{z}]^{T}
\end{align}\

\noindent
である.さらにオイラー法を用いて式(\ref{2-4-9})を離散化すると

\begin{align}
\label{2-4-11}
\textbf{x}(t+1)=\textbf{f}_{t}(\textbf{x}(t))+\textbf{B}_{t}\textbf{v}
\end{align}\

\noindent
ただし

\begin{align}
\label{2-4-12}
\textbf{f}_{t}(\textbf{x}(t))=\textbf{x}(t)+\textbf{f}(\textbf{x}(t))\Delta t, \quad \textbf{B}_{t}=\textbf{B}\Delta t
\end{align}\

\noindent
である.ここで$\Delta t$はサンプリング周期である.

次に観測方程式について見ていく.観測値はUWBモジュールから得られる距離データ,加速度センサデータ,角速度センサデータ,距離センサデータ及びフライトコントローラであるPixracerが推定したヨー角($\psi$)をそれぞれ採用した.
まず,加速度センサ値を用いた角度の算出である.加速度おセンサから得られるセンサ値$\textbf{a}_\textbf{b}$と重力ベクトル$\textbf{g}_\textbf{r}$の関係は回転行列$\textbf{R}$を用いて次のようになる.

\begin{align}
\label{2-4-13}
\begin{split}
\left[
    \begin{array}{c}
      a_{x} \\
      a_{y} \\
      a_{z}
    \end{array}
\right]
&=
\left[
    \begin{array}{ccc}
      \cos\theta\cos\psi & \cos\theta\sin\psi & -\sin\theta\\
      \sin\phi\sin\theta\cos\psi & \sin\phi\sin\theta\sin\psi+\cos\phi\cos\psi & \sin\phi\cos\theta \\
      \cos\phi\sin\theta\cos\psi & \cos\phi\sin\theta\sin\psi-\sin\phi\cos\psi & \cos\phi\cos\theta
    \end{array}
\right]
\left[
    \begin{array}{c}
      0\\
      0 \\
      -g
    \end{array}
\right]\\
&=
\left[
    \begin{array}{c}
      g\sin\theta\\
      -g\cos\theta\sin\phi\\
      -g\cos\theta\cos\phi
    \end{array}
\right]
\end{split}
\end{align}\

式(\ref{2-4-13})の最終項を見ると,$\psi$つまりヨー角成分が現れていないことがわかる.つまり加速度センサのみではヨー角を推定することはできず,そのため今回はヨー角の推定に角速度センサの値とPixracerから得られるヨー角をミキシングしている.ここで観測行列を$\textbf{y}(t) = [a_x, a_y, a_z, d_1, d_2, d_3, d_4, d_h, \psi_{pix}]^T$とおき,式(\ref{2-4-1}),(\ref{2-4-13})と合わせると


\begin{align}
\label{2-4-14}
\textbf{y}(t)=\textbf{h}_{t}(\textbf{x}(t))
\end{align}

\noindent
ただし$\textbf{h}(\cdot)$は

\begin{align} 
\label{2-4-15}
\textbf{h}_{t}(\textbf{x}(t))
=
\left[
    \begin{array}{c}
       g\sin\theta\\
      -g\cos\theta\sin\phi\\
      -g\cos\theta\cos\phi\\
      \sqrt{(x-x_{1})^{2}+(y-y_{1})^{2}+(z-z_{1})^{2}} \\
      \sqrt{(x-x_{2})^{2}+(y-y_{2})^{2}+(z-z_{2})^{2}} \\
      \sqrt{(x-x_{3})^{2}+(y-y_{3})^{2}+(z-z_{3})^{2}} \\
      \sqrt{(x-x_{4})^{2}+(y-y_{4})^{2}+(z-z_{4})^{2}} \\
	z \\
	\psi
    \end{array}
\right]
\end{align}\

\noindent
である.なお$d_h$は距離センサから得られた地面までの距離,$\psi_{pix}$はPixracerが推定したヨー角の値である.式(\ref{2-4-15})に加速度センサ,UWBモジュール,距離センサ,Pixracerからの出力に含まれるノイズ$[\delta a_{x}, \delta a_{y}, \delta a_{z}]^{T},\ [\delta d_1, \delta d_2, \delta d_3, \delta d_4]^{T},\ \delta d_{h},\ \delta \psi_{pix}$を考慮すると観測方程式は


\begin{align}
\label{2-4-16}
\textbf{y}(t)=\textbf{h}_{t}(\textbf{x}(t))+\textbf{w}
\end{align}\

\noindent
となる.ただし$\textbf{w}$は

\begin{align}
\label{2-4-17}
\textbf{w}=[\delta a_{x}, \delta a_{y}, \delta a_{z}, \delta d_1, \delta d_2, \delta d_3, \delta d_4, \delta d_{h}, \delta \psi_{pix}]^{T}
\end{align}

\noindent
以上より状態方程式(\ref{2-4-11})及び観測方程式(\ref{2-4-16})が導出できた.この離散プロセスモデルを2章第1節にて示した拡張カルマンフィルタに対して適用することで位置推定アルゴリズムを得ることができる.





\subsection{従来の構成にオプティカルフローセンサも加えた位置推定システム(複数センサの統合環境)}
ここではオプティカルフローセンサの値も推定アルゴリズムに組み込んだ手法について説明する.次のFig. \ref{system_2}に前のセンサ構成(Fig. \ref{system})にオプティカルフローセンサボードも加えた構成及びFig. \ref{system_core_design2}にオプティカルフローセンサも組み込んだブロック線図をを示す.

\begin{figure}[H]
\begin{center}
\includegraphics[height=100mm]{./Fig2/system_2.png}
\caption{Configuration of sensors with optical flow sensor board}
\label{system_2}
\end{center}
\end{figure}

\begin{figure}[H]
\begin{center}
\includegraphics[height=100mm]{./Fig2/system_core_design2.png}
\caption{Block diagran of positon control system}
\label{system_core_design2}
\end{center}
\end{figure}

\noindent
図からわかるようにオプティカルフローセンサボードとマイコン(Teensy LC)が新たに加わった.ここで,なぜマイコンを二つ(Arduino Pro Mini,Teensy LC)用いているかの理由を説明する.UWBモジュール(DWM1000)及びオプティカルフローセンサ(PMW3901)はSPI通信を用いてマイコンと通信する.SPI通信はマスタ側とスレイブ側が4線で通信する通信規格であり,スレイブの数が複数あっても1つのマイコンでそれらを処理することが可能である.したがって,Fig. \ref{system_2}に示すArduino Pro Mini1つで3つのセンサ(PMW3901, VL53L0X, DWM1000)の値を処理することを考えた.しかし,実際はPMW3901とDWM1000の2つのセンサ間の相性が悪く,同時に2つのセンサの値を取得することが不可能であった.したがって,オプティカルフローセンサボードの値を処理するマイコン(Teensy LC)とUWBモジュールの値を処理するマイコン(Arduino Pro Mini)を分離し,別々に値を取得してから1つのマイコン(Teensy LC)に値をまとめ,UART通信でRaspberry Piに送る構成を組んだ.また,Teensy LCであるが,このマイコンはArduino Pro Miniと異なり,1つのマイコンで複数のUART通信を行うことが可能であり,図のようにArduino Pro Mini及びNAVIO2の2種類のボードと通信を行う必要があったためこのマイコンを採用するに至った.

次にオプティカルフローセンサの値も拡張カルマンフィルタを組み込んだ式を示していく.このアルゴリズムにおいても状態方程式は式(\ref{2-4-11})と同じものを用いる.つまり,状態方程式側にて,加速度センサの積分値を用いてクアッドロータの変位を推定し,角速度センサの値を用いて姿勢角を推定する.観測方程式であるが加速度センサ値,UWBモジュールから得た値,距離センサ値,Pixracerから得た値を用いることは変わらず,そこにオプティカルフローセンサから得た速度値が加わった.つまり観測行列は$\textbf{y}(t) = [a_x, a_y, a_z, d_1, d_2, d_3, d_4, d_h, \psi_{pix}, v_{rx}, v_{ry}]^T$と表される.ここで$v_{rx}$と$v_{ry}$は式(\ref{2-3-1})で表されるヨー角回転を補正した値である.そして,観測方程式は


\begin{align}
\label{2-4-18}
\textbf{y}(t)=\textbf{h}_{t}(\textbf{x}(t))
\end{align}\

\noindent
となり,$\textbf{h}(\cdot)$は

\begin{align} 
\label{2-4-19}
\textbf{h}_{t}(\textbf{x}(t))
=
\left[
    \begin{array}{c}
       g\sin\theta\\
      -g\cos\theta\sin\phi\\
      -g\cos\theta\cos\phi\\
      \sqrt{(x-x_{1})^{2}+(y-y_{1})^{2}+(z-z_{1})^{2}} \\
      \sqrt{(x-x_{2})^{2}+(y-y_{2})^{2}+(z-z_{2})^{2}} \\
      \sqrt{(x-x_{3})^{2}+(y-y_{3})^{2}+(z-z_{3})^{2}} \\
      \sqrt{(x-x_{4})^{2}+(y-y_{4})^{2}+(z-z_{4})^{2}} \\
	z \\
	\psi \\
	v_x \\
	v_y
    \end{array}
\right]
\end{align}\

\noindent
である.前と同様に各種センサから得られるノイズを考慮すると観測方程式は

\begin{align}
\label{2-4-20}
\textbf{y}(t)=\textbf{h}_{t}(\textbf{x}(t))+\textbf{w}
\end{align}\

\noindent
となり,$\textbf{w}$は

\begin{align}
\label{2-4-21}
\textbf{w}=[\delta a_{x}, \delta a_{y}, \delta a_{z}, \delta d_1, \delta d_2, \delta d_3, \delta d_4, \delta d_{h}, \delta \psi_{pix}, \delta v_{rx}, \delta v_{ry}]^{T}
\end{align}

\noindent
である.以上より新たな構成においても観測方程式が導出できた.これら状態方程式,観測方程式を前節と同様に拡張カルマンフィルタに適用することでオプティカルフローセンサも組み込んだ位置推定アルゴリズムを得ることができる.



\newpage


%%%%%%%%%%%%%%%%%%%%%%%%%%%%%%%%%%%%%%%%%%%%%%%%%%%%%%%%%%%%%%%%%%%%%%%%%%%%%%%%%%%%%%%
%%%%%%%%%%%%%%%%%%%%%% 位置制御用クアッドロータの開発 %%%%%%%%%%%%%%%
%%%%%%%%%%%%%%%%%%%%%%%%%%%%%%%%%%%%%%%%%%%%%%%%%%%%%%%%%%%%%%%%%%%%%%%%%%%%%%%%%%%%%%%%%%

\section{位置制御用クアッドロータの開発}





\section{位置制御アルゴリズムの評価}
前章までは各種センサを用いたクアッドロータの位置推定方法について述べてきた.この説では実際にクアッドロータを飛行させ,提案する位置制御手法の精度を見ていく.

\subsection{位置制御手法}
まず,この節ではFig. \ref{system_core_design}及びFig. \ref{system_core_design2}にある目標座標と推定座標の偏差からクアッドロータの位置を制御する手法について述べる.これから述べる手法は[][][]などの手法を参考にした.まず,推定した座標と目標座標との偏差より仮想入力の値を次のように設定する.

\begin{align}
\label{5-1-1}
&U_{x} = k_{px}(x_d - \hat{x} ) + k_{dx}(\dot{x}_d - \dot{\hat{x}} ) + k_{ix} \int_0^t (x_d - \hat{x} ) d\tau  \\
&U_{y} = k_{py}(y_d - \hat{y} ) + k_{dy}(\dot{y}_d - \dot{\hat{y}} ) + k_{iy} \int_0^t (y_d - \hat{y} ) d\tau  
\end{align}

\noindent
$x$軸,$y$軸はそれぞれ機体座標系ではなく,ワールド座標系における軸を表している.式(\ref{5-1-1})からわかるように仮想入力の値は偏差を用いたPIDコントローラより計算できる.これらの入力値を用いると,ピッチ角($\phi$)及びロール角($\theta$)の目標値は以下の式により計算できる.

\begin{align}
\label{5-1-2}
&\phi^{des} = \frac{1}{g} (U_x \sin \psi - U_y  \cos \psi)  \\
&\theta^{des} = \frac{1}{g} (U_x \cos \psi + U_y  \sin \psi).
\end{align}

\noindent
これらの得た値をフライトコントローラであるPixracerが受け取れるPWM信号に変換した後,コントローラに送信することで,クアッドロータの各ロータの回転数が制御され,姿勢角が追従するという流れである.またヨー角($\psi$)に関しては,飛行開始時に置いたクアッドロータの機首が向いている方向を目標角度とし,推定した現在角度との偏差において単純なPID制御器を組むことにより機首の向きが回転しないように制御を行う.


\subsection{ステレオカメラを用いた位置測定システム}
クアッドロータの位置を制御した場合,位置が精度良く制御できているかを確認するため,それの指標として位置の真値が必要である.飛行体や地上移動機に関わらず,移動型のロボットの真値を計測するにはFig. \ref{VICON}に示すVICON[]などといったモーションキャプチャのシステムを用いるのが多い.しかし,こういったシステムは機材だけで数百万円の価格であり,設置費を含むと数千万円に昇ることもこともある.また,Fig. \ref{VICON_room}のように一度室内に環境構築を行うと,システムを他の部屋に移動させることが困難である.したがって,価格がVICONよりも安価であり,必要に応じてシステムを移動させられることなどの利点があることから本論文ではステレオカメラを用いた位置の計測システムを構築した.Fig. \ref{camera_with_chess}に環境構築したステレオカメラシステムを示す.

\begin{figure}[H]
  \begin{center}
    \begin{tabular}{c}

      \begin{minipage}{0.4\hsize}
        \begin{center}
         \includegraphics[height=40mm]{./Fig2/VICON.jpg}
	  \caption{VICON[]}
        \label{VICON}
        \end{center}
      \end{minipage}

      \begin{minipage}{0.6\hsize}
        \begin{center}
          \includegraphics[height=50mm]{./Fig2/VICON_room.jpg}
          \caption{The room where VICON is installed[]}
          \label{VICON_room}
        \end{center}
      \end{minipage}

    \end{tabular}
  \end{center}
\end{figure}


\begin{figure}[H]
\begin{center}
\includegraphics[height=70mm]{./Fig2/camera_with_chess.png}
\caption{Installed stereo camera and chessboard}
\label{camera_with_chess}
\end{center}
\end{figure}






クアッドロータの3次元空間における運動を評価する際,クアッドロータの位置を精度よく測定する必要がある.そこで高速度カメラを2つ用いたステレオカメラによる位置測定システムを開発した.Fig. 28に開発したステレオカメラシステムを示す.ステレオカメラシステムは2つの高速度カメラとカメラで撮影した動画を記録するPCで構成されている.高速度カメラはPoint Grey社のGtasshoppper3(GS3-U3-32S4C-C)を使用した.使用した高速度カメラをFig. 29に示す.また詳細な仕様をTable 6に示す.

\begin{figure}[H]
\begin{center}
\includegraphics[height=60mm]{./Fig/Overview_of_the_stereo_camera_system.png}
\caption{Overview of the stereo camera system.}
\label{Overview_of_the_stereo_camera_system}
\end{center}
\end{figure}

\begin{figure}[H]
\begin{center}
\includegraphics[height=35mm]{./Fig/High-speed_camera_(GS3-U3-32S4C-C).png}
\caption{High-speed camera (GS3-U3-32S4C-C).}
\label{High-speed_camera_(GS3-U3-32S4C-C)}
\end{center}
\end{figure}

\begin{table}[H]
 \caption{High-speed camera specifications.}
 \label{Quad-rotor  specifications}
 \begin{center}
  \begin{tabular}{|c||c|}
   \hline
   Maximum resolution & 2048$\times$4536 \\ 
   \hline
   Maximum frame rate [fps]& 121 \\
   \hline
   Image sensor& Sony IMX252 \\
   \hline
   Sensor type& CMOS \\
   \hline
   Shutter type&Global shutter\\
   \hline
   Angle of view [deg]& 60 \\
   \hline
   Interface & USB3.1 Gen 1 \\
   \hline
  \end{tabular}
 \end{center}
\end{table}

\noindent
Grasshopper3は最大解像度2048$\times$1536,最大フレームレートが121 fpsであるが,今回はPCへの転送速度を考慮し解像度は1024$\times$768,フレームレートは100 fpsで撮影を行う.またGtasshoppper3のインターフェースはUSB3.1で動画記録用PCと接続されている.

今回開発したステレオカメラシステムのために赤外線LEDを搭載したマーカーを製作した.Fig. 30に製作したマーカーを示す.このマーカーを赤外線透過フィルタを通して撮影すると,Fig. 31のようになりマーカー位置を特定しやすくなる.マーカーは小型でありクアッドロータに搭載することができるため,クアッドロータの飛行中における位置の計測が可能になる.またLEDの点灯はリレーにより制御できるため,点灯のタイミングを合わせることで計測データの同期をとることができる.

\begin{figure}[htbp]
 \begin{minipage}{0.5\hsize}
  \begin{center}
   \includegraphics[width=60mm]{./Fig/Infrared_LED_marker.png}
  \end{center}
  \caption{Infrared LED marker.}
  \label{Infrared_LED_marker}
 \end{minipage}
 \begin{minipage}{0.5\hsize}
  \begin{center}
   \includegraphics[width=60mm]{./Fig/Image_obtained_by_photographing_infrared_LED_marker_through_infrared_longpss_filter.png}
  \end{center}
  \caption{Image obtained by photographing infrared LED marker through infrared longpass filter.}
  \label{Image_obtained_by_photographing_infrared_LED_marker_through_infrared_longpss_filter}
 \end{minipage}
\end{figure}

\noindent
また赤外線透過フィルタは富士フィルムのIRフィルター(IR-80)を使用した(Fig. 32)

\begin{figure}[H]
\begin{center}
\includegraphics[height=50mm]{./Fig/Infrared_longpass_filter_(IR-80).png}
\caption{Infrared longpass filter (IR-80).}
\label{Infrared_longpass_filter_(IR-80)}
\end{center}
\end{figure}

\newpage

\subsection{ステレオカメラの原理}
ステレオカメラによる位置測定システムのため,カメラを数式モデルで表現する.今回カメラモデルとして最も単純なカメラモデルであるピンホールカメラモデルを採用する\cite{k}.ピンホールカメラモデルをFig. 33に示す.

\begin{figure}[H]
\begin{center}
\includegraphics[height=40mm]{./Fig/Pinhole_camera_model.png}
\caption{Pinhole camera model.}
\label{Pinhole_camera_model}
\end{center}
\end{figure}

\noindent
理想的なピンホールカメラではピンホールから画像平面までの距離は焦点距離と完全に一致する.$f$はカメラの焦点距離,$z$はカメラから対象物までの距離,$T$は物体の長さ,$t$は画像平面上での対象物の長さを示す.三角形の相似より$T$と$t$の関係は次式で示される.

\begin{align}
-t=f\frac{T}{Z}
\end{align}\

ここで計算をより簡単にするためにFig. 34に示すようにピンホール平面と画像平面の位置を入れ替える.

\begin{figure}[H]
\begin{center}
\includegraphics[height=40mm]{./Fig/Mathmatically_equivalent_camera_model.png}
\caption{Mathematically equivalent camera model.}
\label{Mathmatically_equivalent_camera_model}
\end{center}
\end{figure}

\noindent
このカメラモデルでは物理世界にある点$Q=(X,Y,Z)$が画像平面上のピクセル座標$(x,y)$に投影される.その時点$Q$と点$q$の関係は次式で表される.

\begin{align}
x=f_{x}(\frac{X}{Z})+c_{x},y=f_{x}(\frac{Y}{Z})+c_{y}
\end{align}\

ここでは$c_{x},c{y}$ [pixel]を導入し画像平面の中心と光軸のずれ量を表す.また2つの異なる焦点距離$f_{x},f_{y}$ [pixel]を導入している.その理由は画像素子の各ピクセルは正確な正方形ではなく実際は長方形だからである.焦点距離$f_{x},f_{y}$はレンズの物理的な焦点距離$F$ [mm]と各受光素子の大きさ$s_{x},s_{y}$ [pixel/mm]の積となる.カメラ単体のキャリブレーションにおいて測定できるのは$c_{x},c_{y},f_{x},f_{y}$の4つのパラメータであり,$F$や$s_{x},s_{y}$は直接的に観測できない.また,この4つのパラメータのことを内部パラメータと呼ぶ.
次にレンズ歪みを数式モデルで表す.現実世界のカメラのレンズは完璧ではなくレンズ歪みが発生してしまう.レンズ歪みは主に2つに分けられ,レンズの形状に起因する半径方向歪みとカメラの組み立ての不具合に起因する円周方向歪みに分けられる.円周方向歪みの場合,画像素子の中心において歪みは0であり,中心から離れるにつれ,歪みが大きくなる.したがって半径方向歪みは3つのパラメータ$k_{1},k_{2},k_{3}$を用いて次式のように表される.

\begin{align}
x_{corrected}=x(1+k_{1}r^{2}+k_{2}r^{4}+k_{3}r^{6})
\end{align}
\begin{align}
y_{corrected}=y(1+k_{1}r^{2}+k_{2}r^{4}+k_{3}r^{6})
\end{align}\

\noindent
ここで$(x,y)$は画像平面における歪みを補正する前のもとの位置であり,$(x_{correted},y_{corrected})$は補正の結果得られる補正後の位置である.円周方向歪みはレンズが画像平面に対して完全に平行に取り付けられていないことが原因で起こる.そして2つのパラメータ$p_{1},p_{2}$を用いて次式のように表すことができる.

\begin{align}
x_{corrected}=x+(2p_{1}xy+p_{2}(r^{2}+2x^{2}))
\end{align}
\begin{align}
y_{corrected}=y+(p_{1}(r^{2}+2y^{2})+2p_{2}xy)
\end{align}\

\noindent
以上からカメラを数式モデルで扱う際には4つの内部パラメータ($c_{x},c_{y},f_{x},f_{y}$)と5つの歪みパラメータ($k_{1},k_{2},k_{3},p_{1},p_{2}$)が必要になる.

\begin{figure}[H]
\begin{center}
\includegraphics[height=60mm]{./Fig/Difinition_of_stereo_camera_corrdinates.png}
\caption{Definition of stereo camera coordinates.}
\label{Difinition_of_stereo_camera_corrdinates}
\end{center}
\end{figure}

最後にステレオカメラを用いた位置推定方法の原理を説明する.まず2つのカメラをFig. 35に示すように配置する.ここで2つのカメラの画像平面が同一平面上にあり,光軸が平行であり,同じ焦点距離$f$を持つとする.今回左側のカメラの光学中心がカメラ座標の原点となるように配置した.Fig. 35のように2つのカメラで同じ点を観測した時$x$方向における視差は$d=x_{l}-x_{r}$で定義される.視差$d$を用いると三角形の相似より以下の式が求まる.

\begin{align}
\frac{L-d}{z_{q}-f}=\frac{L-(x_{l}-x_{r})}{z_{q}-f}=\frac{L}{z_{q}}
\end{align}\

\noindent
この式を変形すると

\begin{align}
z_{q}=\frac{fL}{x_{l}-x_{r}}
\end{align}\

\noindent
となり,点Qの$z$座標が求まる.これを$x_{q},y_{q}$についても同じように求めると

\begin{align}
x_{q}=\frac{x_{l}L}{x_{l}-x_{r}},y_{q}=\frac{y_{l}L}{x_{l}-x_{r}}=\frac{y_{r}L}{x_{l}-x_{r}}
\end{align}\

\noindent
となる.以上より式(53)と式(54)から点Qの位置$(X,Y,Z)$を求めることができる.

しかし実際に2つのカメラの画像平面を正確に同一平面上に置き,光軸を正確に平行にすることは困難である.そこで2つのカメラを用いてキャリブレーションを行い,2つのカメラの相対的な位置関係を示す並進ベクトルと回転行列を求め,そしてその並進ベクトルと回転行列を用いて画像処理を行い2つのカメラの平行化を行う.画像処理により平行化を行うことで式(53)と式(54)の関係が成り立つため,任意の点の位置を推定することができる.


\subsection{ステレオカメラのキャリブレーション}

\begin{figure}[H]
\begin{center}
\includegraphics[height=50mm]{./Fig/Chessboard.png}
\caption{Chessboard.}
\label{Chessboard}
\end{center}
\end{figure}


Fig. 36に示すようなチェスボードを用いてステレオカメラのキャリブレーションを行った.チェスボードの正方形の一辺の長さは33.0 mmである.そしてこのチェスボードを様々な向きに回転させステレオカメラで撮影を行った.実際にキャリブレーションに使用した15組の画像をFig. 37とFig. 38に示す.このキャリブレーションにより,左右のカメラの内部パラメータと外部パラメータ.及び左右のカメラの位置関係を示す並進ベクトル$\textbf{T}_{stereo}$と回転行列$\textbf{R}_{stereo}$を求めることができる.式(55)~(58)に算出した左右のカメラの内部パラメータと歪みパラメータを示す.

\begin{align}
[c_{x},c_{y},f_{x},f_{y}]_{left}=[521.5,413.3,1157.7,1161.9]
\end{align}\

\begin{align}
[c_{x},c_{y},f_{x},f_{y}]_{right}=[505.8,398.3,1168.4,1170.4]
\end{align}\

\begin{align}
[k_{1},k_{2},k_{3},p_{1},p_{1}]_{left}=[-0.317,1.099,0.003,0.003,0.000]
\end{align}\

\begin{align}
[k_{1},k_{2},k_{3},p_{1},p_{1}]_{right}=[-0.189,0.143,-0.000,-0.005,0.000]
\end{align}\

\noindent
また式(59)と式(60)に並進ベクトル$\textbf{T}_{stereo}$と回転行列$\textbf{R}_{stereo}$を示す.

\begin{align}
\textbf{T}_{stereo}=
\left[
    \begin{array}{c}
     679.91 \\
      3.10 \\
      28.04 
    \end{array}
\right]
\end{align}\

\begin{align}
\textbf{R}_{stereo}=
\left[
    \begin{array}{ccc}
      1.000  & 0.000  & 0.028\\
      0.000  & 1.000  & -0.003\\
      -0.028 & 0.003 & 1.000
    \end{array}
\right]
\end{align}\

\noindent
キャリブレーションで得られたステレオカメラのパラメータをもとにマーカー位置の測定を行っていく.

\newpage
\begin{figure}[H]
\begin{center}
\includegraphics[height=190mm]{./Fig/Stereo_camera_calibration_(left).png}
\caption{Stereo camera calibration (left).}
\label{Stereo_camera_calibration_(left)}
\end{center}
\end{figure}
\newpage
\begin{figure}[H]
\begin{center}
\includegraphics[height=190mm]{./Fig/Stereo_camera_calibration_(right).png}
\caption{Stereo camera calibration (right).}
\label{Stereo_camera_calibration_(right)}
\end{center}
\end{figure}
\newpage

\subsection{ステレオカメラの測定範囲の設定}
ステレオカメラにおいて,計測精度は画像中の一画素が実際にどのくらいの長さに対応するかで決まる.つまり,レンズ・カメラの解像度・撮影距離・カメラ間距離によって精度は変化する.ステレオカメラにおける奥行き誤差$\delta z$を模式的に表したものをFig. 39に示す.
\begin{figure}[H]
\begin{center}
\includegraphics[height=70mm]{./Fig/Error_in_the_depth_direction_of_the_stereo_camera.png}
\caption{Error in the depth direction of the stereo camera.}
\label{Error_in_the_depth_direction_of_the_stereo_camera}
\end{center}
\end{figure}

\noindent
$\delta z$を数式で表すと幾何学的関係より次式のようになる.

\begin{align}
\delta z =\frac{L}{2\tan (\alpha - \beta)}-l
\end{align}\

\noindent
ただし

\begin{align}
\alpha = \arctan\frac{L}{2l}
\end{align}\

\noindent
である.奥行き方向の誤差が0.05 m以内になるように,式(61)に基づいてFig. 40に示すような計測範囲を設定した.


\begin{figure}[H]
\begin{center}
\includegraphics[height=80mm]{./Fig/Measurement_range_of_stereo_camera.png}
\caption{Measurement range of stereo camera.}
\label{Measurement_range_of_stereo_camera}
\end{center}
\end{figure}

\newpage
%%%%%%%%%%%%%%%%%%%%%%%%%%%%%%%%%%%%%%%%%%%%%%%%%%%%%%%%%%%%%%%%%%%%%%%%%%%%%%%%%%%%%%%
%%%%%%%%%%%%%%%%%%%%%% 姿勢制御システム %%%%%%%%%%%%%%%
%%%%%%%%%%%%%%%%%%%%%%%%%%%%%%%%%%%%%%%%%%%%%%%%%%%%%%%%%%%%%%%%%%%%%%%%%%%%%%%%%%%%%%%%%%

\section{姿勢制御システム}

\begin{figure}[H]
\begin{center}
\includegraphics[height=40mm]{./Fig/Block_diagram_of_the_attitude_control_system.png}
\caption{Block diagram of the attitude control system.}
\label{Block_diagram_of_the_attitude_control_system}
\end{center}
\end{figure}

Fig. 41に制御システムのブロック線図を示す.このブロック線図において姿勢角推定を拡張カルマンフィルタを使って行い,姿勢制御にIntegral Backstepping制御を用いる.姿勢角推定に関しては3軸加速度センサ,3軸ジャイロセンサ及び3軸地磁気センサを組み合わせ姿勢角の推定する.以下ではまず拡張カルマンフィルタを用いた姿勢角推定法について述べた後,Integral Backstepping制御を用いた姿勢制御法について述べる.

\subsection{姿勢推定}

\subsubsection{磁気センサのキャリブレーション}
地磁気センサは地球の地磁気を検知するセンサである.一般的にロボットや携帯電話では地磁気により地球の地磁気を検知することで,方位を測定している.地磁気センサは主に2軸のものと3軸のものに分けられるが,今回は3軸の地磁気センサに焦点を当てる.
地磁気センサを様々な方向に回転させるとx,y,z軸方向の地磁気データは式(63)で表される二次曲面を描くような形でプロットされる.

\begin{align}
ax^{2}+by^{2}+cz^{2}+2fyz+2gxz+2hxy+2px+2qy+2rz+d=0
\end{align}\

\noindent
式(64)を行列式で表すと次のようになる.

\begin{align}
\textbf{x}^{T}\textbf{M}\textbf{x}+\textbf{x}^{T}\textbf{n}+d=0
\end{align}

\noindent
ただし

\begin{align}
\textbf{x}=
\left[
    \begin{array}{c}
      x \\
      y \\
      z 
    \end{array}
\right],
\textbf{M}=
\left[
    \begin{array}{ccc}
      a & f & g\\
      f & b & h \\
      g & h & c
    \end{array}
\right],
\textbf{n}=
\left[
    \begin{array}{c}
      p \\
      q \\
      r 
    \end{array}
\right]
\end{align}\

\noindent
周囲に磁気物体などがない理想的な場合$\textbf{M},\textbf{n},d$は

\begin{align}
\textbf{M}=
\left[
    \begin{array}{ccc}
      1 & 0 & 0\\
      0 & 1 & 0 \\
      0 & 0 & 1
    \end{array}
\right],
\textbf{n}=
\left[
    \begin{array}{c}
      0 \\
      0 \\
      0 
    \end{array}
\right],
d=F^{2}
\end{align}\

\noindent
である.ここで$F$は地磁気の強度を表す.この時,式(63)は原点中心,半径$F$の球の方程式になる.
しかし,現実世界ではセンサの測定誤差と周囲の磁性体の磁気干渉により,球の中心が原点からずれたり,球が歪み楕円体になったりする.それゆえ地磁気センサから得られたデータをそのまま用いて姿勢角を求めた際,正しい姿勢角を求めることができない.そこでセンサの測定誤差や磁気干渉をモデル化し,原点中心の球に変換するための変換式を求める必要がある\cite{l}.

\begin{figure}[H]
\begin{center}
\includegraphics[height=40mm]{./Fig/Schematic_representation_of_each_instrumentation_error.png}
\caption{Schematic representation of each instrumentation error.}
\label{Schematic_representation_of_each_instrumentation_error}
\end{center}
\end{figure}

まずセンサの測定誤差に関して説明する.測定誤差はセンサごとに異なり,一定である.そして測定誤差は各軸のスケール誤差(Fig. 42-(a)),センサ軸の非直交性(Fig. 42-(b)),オフセット(Fig. 42-(c))の3つに起因する.
各軸のスケール誤差は対角行列を用いて次のように表される.

\begin{align}
\textbf{S}=
\left[
    \begin{array}{ccc}
      s_{x} & 0 & 0\\
      0 & s_{y} & 0 \\
      0 & 0 & s_{z}
    \end{array}
\right]
\end{align}\

\noindent
センサ軸の非直交性は3次元の正規直交基底ベクトル$\textbf{n}_{x},\textbf{n}_{y},\textbf{n}_{z}$を用いて次のように表される.

\begin{align}
\textbf{S}=
\left[
    \begin{array}{ccc}
      \textbf{n}_{x} & \textbf{n}_{y} & \textbf{n}_{z}\\
    \end{array}
\right]
\end{align}\

\noindent
ここで式(68)の行列の各列は各センサ軸に対応する.最後にセンサーのオフセット量は次のように表される.

\begin{align}
\textbf{b}_{SO}=
\left[
    \begin{array}{ccc}
    b_{SO_{x}} & b_{SO_{y}} & b_{SO_{z}}\\
    \end{array}
\right]^{T}
\end{align}\

\begin{figure}[H]
\begin{center}
\includegraphics[height=40mm]{./Fig/Schematic_representation_of_each_magnetic_interference.png}
\caption{Schematic representation of each magnetic interference.}
\label{Schematic_representation_of_each_magnetic_interference}
\end{center}
\end{figure}

次に磁気干渉について説明する.磁気干渉は,センサの周囲に存在する強磁性要素によって引き起こされ,永久磁石や磁化された鉄の残留磁気により地磁気にオフセットが生じたり(Fig. 43-(a), hard iron),外部磁場と強磁性体の相互作用によって磁場の強度や方向が変化(Fig. 43-(b), soft iron)したりする.前者は

\begin{align}
\textbf{b}_{hi}=
\left[
    \begin{array}{ccc}
    b_{hi_{x}} & b_{hi_{y}} & b_{hi_{z}}\\
    \end{array}
\right]^{T}
\end{align}\

\noindent
で表され,後者は

\begin{align}
\textbf{S}=
\left[
    \begin{array}{ccc}
      a_{11} & a_{12} & a_{13}\\
      a_{21} & a_{22} & a_{23} \\
      a_{31} & a_{32} & a_{33}
    \end{array}
\right]
\end{align}\

\noindent
で表される.式(67)~(71)より,センサーから得られる磁場$\textbf{h}_{m}$は次のように表すことができる.

\begin{align}
\textbf{h}_{m}=\textbf{SN}(\textbf{A}_{si}\textbf{h}+\textbf{h}_{hi})+\textbf{b}_{SO}
\end{align}\

\noindent
ただし$\textbf{h}$は測定誤差や磁気干渉が無い真の磁場ベクトルである.式(72)のそれぞれの係数を求めるのは困難なため,次式のように簡略化する.

\begin{align}
\textbf{h}_{m}=\textbf{A}_{h}\textbf{h}+\textbf{b}_{h}
\end{align}\

\noindent
ただし

\begin{align}
\textbf{A}_{h}=\textbf{SN}\textbf{A}_{si},\textbf{b}_{h}=\textbf{SNb}_{hi}+\textbf{b}_{SO}
\end{align}\

\noindent
である.また磁場ベクトル$\textbf{h}$について解くと

\begin{align}
\textbf{h}=\textbf{A}_{h}^{-1}(\textbf{h}_{m}-\textbf{b}_{h})
\end{align}\

\noindent
となる.ここで$\textbf{h}$のノルムは地磁気の強度$F$であるから

\begin{align}
\textbf{h}^{T}\textbf{h}=F^{2}
\end{align}\

\noindent
式(76)に式(75)を代入すると

\begin{align}
\textbf{h}_{m}^{T}\textbf{A}_{h}^{-T}\textbf{A}_{h}^{-1}\textbf{h}_{m}-2\textbf{h}_{m}^{T}\textbf{A}_{h}^{-1}\textbf{b}+\textbf{b}^{T}\textbf{A}_{h}^{-T}\textbf{A}_{h}^{-1}\textbf{b}-F^{2}=0
\end{align}\

\noindent
となり,ここで

\begin{align}
\textbf{M}=\textbf{A}_{h}^{-T}\textbf{A}_{h}^{-1},\textbf{n}=-2\textbf{A}_{h}^{-1}\textbf{b},d=\textbf{b}^{T}\textbf{A}_{h}^{-T}\textbf{A}_{h}^{-1}\textbf{b}-F^{2}
\end{align}\

\noindent
とおくと最終的に

\begin{align}
\textbf{h}_{m}^{T}\textbf{M}\textbf{h}_{m}+\textbf{h}_{m}^{T}\textbf{n}+d=0
\end{align}

\noindent
となる.式(79)を見ると式(64)と同じ形であることがわかる.つまり磁気センサの測定データから$\textbf{M},\textbf{n},d$を最小二乗法を用いて推定することで$\textbf{A}_{h}^{-1},d$を求めることができる\cite{m}.$\textbf{A}_{h}^{-1},d$を算出する式は次式のようになる.

\begin{align}
\textbf{A}_{h}^{-1}=\frac{F}{\sqrt{\textbf{n}^{T}\textbf{M}\textbf{n}-d}}\textbf{M}^{\frac{1}{2}},\textbf{b}=-\textbf{M}^{-1}\textbf{n}
\end{align}\

\noindent
求めた$\textbf{A}_{h}^{-1},d$を式(75)に代入することで,生の磁気センサのデータから測定誤差や地磁気干渉が無い真の磁場ベクトルに変換する式を得ることができる.



\subsubsection{拡張カルマンフィルタについて}
3軸加速度センサ,3軸ジャイロセンサ及び3軸地磁気センサによって観測されたシステム及び観測状態はそれぞれシステム雑音及び観測雑音と呼ばれる雑音に影響を受けるので,システムの状態を推定するためには状態推定器が必要とされる.カルマンフィルタは線形化されたシステムの状態を推定するための状態推定器としてよく知られている.しかし3次元空間を自由に飛行するクアッドロータのシステムは非線形システムであり,線形システムに対して有効なカルマンフィルタは使うことができない.そこで今回,非線形システムの状態を推定するために拡張カルマンフィルタを適用する.ここでは最初にカルマンフィルタの導出を行ったあと拡張カルマンフィルタの導出を行う.

まずカルマンフィルタについて説明する.対象とする線形離散時間状態空間モデルが次のように記述されるとする.

\begin{align}
\textbf{x}(t+1)=\textbf{A}\textbf{x}(t)+\textbf{B}\textbf{v}(t)
\end{align}
\begin{align}
\textbf{y}(t)=\textbf{C}\textbf{x}(t)+\textbf{w}(t)
\end{align}\

\noindent
ここで$\textbf{x}(t)$はn次元の状態ベクトルであり,$\textbf{y}(t)$はp次元観測ベクトルである.また$\textbf{v}(t)$は平均値ベクトル0,共分散行列$\textbf{Q}$のr次元システム雑音ベクトル,$\textbf{w}(t)$は平均値ベクトル0,共分散行列$\textbf{R}$のp次元観測雑音ベクトルであり,互いに独立な正規性白色雑音と仮定する.さらに$\textbf{A}$は$n\times n$行列,$\textbf{B}$は$n\times r$行列,$\textbf{C}$は$p\times n$行列である.状態空間モデルの係数行列及びベクトル$\textbf{A},\textbf{B},\textbf{C}$はそれぞれ基地であると仮定する.また時刻$t$における状態$\textbf{x}$の推定値として事前推定値$\hat{\textbf{x}}^{-}(t)$と事後推定値$\hat{\textbf{x}}(t)$の2つの量を定義する\cite{n}.事前推定値とは時刻$t-1$までに利用可能なデータに基づいた,時刻$t$における$\textbf{x}$の予測推定値であり,事後推定値とは$\textbf{y}(t)$を含めた時刻$t$までのデータを用いた$\textbf{x}$のフィルタリング推定値であり,これが時刻$t$において求めるべき推定値である.

\begin{figure}[H]
\begin{center}
\includegraphics[height=40mm]{./Fig/Kalman_filter_overview.png}
\caption{Kalman filter overview.}
\label{Kalman_filter_overview}
\end{center}
\end{figure}

カルマンフィルタでは状態推定値はFig. 44のように時間更新されていく.推定値の更新は予測ステップとフィルタリングステップの2つのプロセスを経て行われる.予測ステップは1時刻前の推定値$\hat{\textbf{x}}$及び誤差の共分散行列$\textbf{P}(t-1)$から現時刻での事前推定値$\hat{\textbf{x}}^{-}(t)$及び事前誤差共分散行列$\textbf{P}^{-}(t)$までを処理し,フィルタリングステップでは$\hat{\textbf{x}}^{-}(t)$及び$\textbf{P}^{-}(t)$から現時刻での事後推定値$\hat{\textbf{x}}(t)$及び事後誤差共分散行列$\textbf{P}(t)$までを処理する.このフィルタリングステップでは時系列モデルと最新の観測値$\textbf{y}(t)$を利用する.予測ステップにおける更新式は

\begin{align}
\hat{\textbf{x}}^{-}(t)=\textbf{A}\hat{\textbf{x}}(t-1)
\end{align}
\begin{align}
\textbf{P}^{-}(t)=\textbf{A}\textbf{P}(t-1)\textbf{A}^{T}+\textbf{B}\textbf{Q}\textbf{B}^{T}
\end{align}\

\noindent
で表される.またフィルタリングステップにおける更新式は

\begin{align}
\hat{\textbf{x}}(t)=\hat{\textbf{x}}^{-}(t)+\textbf{G}(t)(\textbf{y}(t)-\textbf{C}\hat{\textbf{x}}^{-}(t))
\end{align}
\begin{align}
\textbf{P}(t)=(\textbf{I}-\textbf{G}(k)\textbf{C})\textbf{P}^{-}(t)
\end{align}\

\noindent
ただし行列$\textbf{G}$は次の式で算出することができ,カルマンゲインと呼ばれる.カルマンゲイン$\textbf{G}$は$n\times p$行列である.

\begin{align}
\textbf{G}(t)=\textbf{P}^{-}(k)\textbf{C}^{T}(\textbf{C}\textbf{P}^{-}(k)\textbf{C}^{T}+\textbf{R})^{-1}
\end{align}\

\noindent
また共分散行列$\textbf{P}^{-}(t),\textbf{P}(t)$は$n\times n$行列,行列$\textbf{I}$は$n\times n$の単位行列である.

状態推定値の初期値に関しては,状態推定値の初期値に関する事前情報が利用できる場合にはその値を用いればよいが,事前情報がない場合は

\begin{align}
\hat{\textbf{x}}(0)=0
\end{align}\

\noindent
とおくことが多い.また共分散行列の初期値に関しては

\begin{align}
\textbf{P}(0)=\gamma\textbf{I}
\end{align}\

\noindent
とおくことが多い.ただし$\gamma$は調整パラメータである定数で0~1000までの値が用いられる.一般に観測雑音が大きくSN比が悪い場合には$\gamma$を小さく設定した方が良い.

次に拡張カルマンフィルタについて説明する.拡張カルマンフィルタは非線形システムを各時刻においてテイラー級数展開を用いて線形化し,それぞれの時刻において事変カルマンフィルタを適応する方法である.対象とする離散時間非線形状態空間モデルが次のように記述されるとする.

\begin{align}
\textbf{x}(t+1)=\textbf{f}(\textbf{x}(t))+\textbf{B}\textbf{v}(t)
\end{align}
\begin{align}
\textbf{y}(t)=\textbf{h}(\textbf{x}(t))+\textbf{w}(t)
\end{align}\

\noindent
ここで$\textbf{f}(\cdot)$と$\textbf{h}(\cdot)$はベクトル値をとる$\textbf{x}(t)$の非線形関数である.また$\textbf{v}(t)$は平均値ベクトル0,共分散行列$\textbf{Q}$のr次元システム雑音ベクトル,$\textbf{w}(t)$は平均値ベクトル0,共分散行列$\textbf{R}$のp次元観測雑音ベクトルであり,互いに独立な正規性白色雑音と仮定する.
時刻$t,t+1$においてそれぞれ事前状態推定値$\hat{\textbf{x}}^{-}$と事後推定値$\hat{\textbf{x}}$が利用可能であるという仮定の下で式(90)と式(91)の非線形関数をテイラー級数展開を用いて線形近似すると

\begin{align}
\textbf{f}(\textbf{x}(t))=\textbf{f}(\hat{\textbf{x}}(t))+\textbf{A}(t)(\textbf{x}(t)-\hat{\textbf{x}}(t)),\left.\textbf{A}(t)=\frac{\partial \textbf{f}(\textbf{x})}{\partial \textbf{x}}\right|_{\textbf{x}=\hat{\textbf{x}}(t)}
\end{align}
\begin{align}
\textbf{h}(\textbf{x}(t))=\textbf{h}(\hat{\textbf{x}}^{-}(t))+\textbf{c}^{T}(t)(\textbf{x}(t)-\hat{\textbf{x}}^{-}(t)),\left.\textbf{C}^{T}(t)=\frac{\partial \textbf{h}(\textbf{x})}{\partial \textbf{x}}\right|_{\textbf{x}=\hat{\textbf{x}}^{-}(t)}
\end{align}\

\noindent
となる.式(90)と式(91)にそれぞれ式(92)と式(93)に代入すると

\begin{align}
\textbf{x}(t+1)=\textbf{A}(t)\textbf{x}(t)+\textbf{b}\textbf{v}(t)+\textbf{f}(\hat{\textbf{x}}(t))-\textbf{A}(t)\hat{\textbf{x}}(t)
\end{align}
\begin{align}
\textbf{y}(t)=\textbf{C}^{T}(t)\textbf{x}(t)+\textbf{w}(t)+\textbf{h}(\hat{\textbf{x}}^{-}(t))-\textbf{C}^{T}(t)\hat{\textbf{x}}^{-}(t)
\end{align}\

\noindent
となる.さらに

\begin{align}
\textbf{u}(t)=\textbf{f}(\hat{\textbf{x}}(t))-\textbf{A}(t)\hat{\textbf{x}}(t)
\end{align}
\begin{align}
\textbf{z}(t)=\textbf{y}(t)-\textbf{h}(\hat{\textbf{x}}^{-}(t))+\textbf{C}^{T}(t)\hat{\textbf{x}}^{-}(t)
\end{align}\

\noindent
とおくと,式(94)と式(95)はそれぞれ次式のようになる.

\begin{align}
\textbf{x}(t+1)=\textbf{A}\textbf{x}(t)+\textbf{B}\textbf{v}(t)+\textbf{u}(t)
\end{align}
\begin{align}
\textbf{z}(t)=\textbf{C}\textbf{x}(t)+\textbf{w}(t)
\end{align}\

\noindent
このように非線形システムをテイラー級数展開を用いて線形近似すると制御入力がある場合のカルマンフィルタになる.以上より予測ステップにおける更新式は

\begin{align}
\hat{\textbf{x}}^{-}(t)=\textbf{f}(\hat{\textbf{x}}(t-1))
\end{align}
\begin{align}
\textbf{P}^{-}(t)=\textbf{A}(k-1)\textbf{P}(t-1)\textbf{A}^{T}(k-1)+\textbf{B}\textbf{Q}\textbf{B}^{T}
\end{align}\

\noindent
となる.ここで$\textbf{A}(t-1)$は

\begin{align}
\left.\textbf{A}(t-1)=\frac{\partial \textbf{f}(\textbf{x})}{\partial \textbf{x}}\right|_{\textbf{x}=\hat{\textbf{x}}(t)}
\end{align}\

\noindent
となる.またフィルタリングステップの更新式は

\begin{align}
\hat{\textbf{x}}(t)=\hat{\textbf{x}}^{-}(t)+\textbf{g}(t)(\textbf{y}(t)-\textbf{h}(\hat{\textbf{x}}^{-}(t)))
\end{align}
\begin{align}
\textbf{P}(t)=(\textbf{I}-\textbf{G}(k)\textbf{C}^{T}(t))\textbf{P}^{-}(k)
\end{align}\

\noindent
となる.ここでカルマンゲイン$G(t)$及び$\textbf{C}^{T}(t)$は次式で計算することができる.

\begin{align}
\textbf{G}(t)=\textbf{P}^{-}(k)\textbf{C}^{T}(\textbf{C}\textbf{P}^{-}(k)\textbf{C}^{T}+\textbf{R})^{-1}
\end{align}\
\begin{align}
\left.\textbf{C}^{T}(t)=\frac{\partial \textbf{h}(\textbf{x})}{\partial \textbf{x}}\right|_{\textbf{x}=\hat{\textbf{x}}^{-}(t)}
\end{align}\

\noindent
拡張カルマンフィルタにおいては各時刻ごとに$\textbf{f}$と$\textbf{h}$のヤコビアンを求めるため偏微分の計算を行わなければならない.そのため,微分可能な滑らかな非線形性の場合には拡張カルマンフィルタを適用できるが,不連続な非線形性をもつ場合には適用できない.



\subsubsection{姿勢推定アルゴリズム}
ここでは拡張カルマンフィルタを用いた姿勢の推定アルゴリズムの導出を行う.はじめに状態方程式を求める.オイラー角$\Theta=[\phi,\theta,\psi]^{T}$と機体座標における角速度ベクトル$\omega=[p,q,r]^{T}$の関係は式(11)より

\begin{align}
\frac{d}{dt}\Theta = W \omega
\end{align}\

\noindent
ただし

\begin{align}
W=
\left[
    \begin{array}{ccc}
      1 & \sin\phi\tan\theta & \cos\phi\tan\theta\\
      0 & \cos\phi & -\sin\phi \\
      0 & \sin\phi /\cos\theta & \cos\phi / \cos\theta
    \end{array}
\right]
\end{align}\

\noindent
ここで$\omega=[p,q,r]^{T}$は角速度センサから得られた角速度ベクトルである.状態変数$\textbf{x}(t)$を$\textbf{x}(t)=[\phi,\theta,\psi]^{T}$とおき,次式のように変換する.

\begin{align}
\dot{\textbf{x}}(t)=\textbf{f}(\textbf{x}(t))
\end{align}

\noindent
ただし$\textbf{f}(\cdot)$は

\begin{align}
\textbf{f}(\textbf{x}(t))=
\left[
    \begin{array}{c}
      p + p\sin\phi\tan\theta + p\cos\phi\tan\theta \\
      q\cos\phi -q\sin\phi \\
      r\sin\phi /\cos\theta +r\cos\phi / \cos\theta
    \end{array}
\right]
\end{align}\\

\noindent
式(110)に角速度センサからの出力に含まれるノイズ$[\delta p,\delta q,\delta r]$を考慮すると次式のようになる.

\begin{align}
\dot{\textbf{x}}(t)=\textbf{f}(\textbf{x}(t))+\textbf{B}\textbf{v}
\end{align}\

\noindent
ただし$\textbf{B},\textbf{v}$は

\begin{align}
\textbf{B}=
\left[
    \begin{array}{ccc}
      1 & 0 & 0\\
      0 & 1 & 0\\
      0 & 0 & 1
    \end{array}
\right],
\textbf{v}=
\left[
    \begin{array}{c}
      \delta p  \\
      \delta q  \\
      \delta r
    \end{array}
\right]
\end{align}\\

\noindent
さらにオイラー法を用いて式(111)を離散化すると

\begin{align}
\textbf{x}(t+1)=\textbf{f}_{t}(\textbf{x}(t))+\textbf{B}_{t}\textbf{v}
\end{align}\

\noindent
ただし

\begin{align}
\textbf{f}_{t}(\textbf{x}(t))=\textbf{x}(t)+\textbf{f}(\textbf{x}(t))\Delta t,\textbf{B}_{t}=\textbf{B}\Delta t
\end{align}\

\noindent
ここで$\Delta t$はサンプリング周期である.

次に観測方程式を求める.今回の姿勢推定アルゴリズムでは加速度センサ及び地磁気センサを使って角度の補正を行う.加速度センサから得られる3次元加速度データ$[a_{x},a_{y},a_{z}]^T$と重力ベクトル$[0,0,-g]^{T}$の関係は式(5)の回転行列を用いて次のようになる.

\begin{align}
\begin{split}
\left[
    \begin{array}{c}
      a_{x} \\
      a_{y} \\
      a_{z}
    \end{array}
\right]
&=
\left[
    \begin{array}{ccc}
      \cos\theta\cos\psi & \cos\theta\sin\psi & -\sin\theta\\
      \sin\phi\sin\theta\cos\psi & \sin\phi\sin\theta\sin\psi+\cos\phi\cos\psi & \sin\phi\cos\theta \\
      \cos\phi\sin\theta\cos\psi & \cos\phi\sin\theta\sin\psi-\sin\phi\cos\psi & \cos\phi\cos\theta
    \end{array}
\right]
\left[
    \begin{array}{c}
      0\\
      0 \\
      -g
    \end{array}
\right]\\
&=
\left[
    \begin{array}{c}
      g\sin\theta\\
      -g\cos\theta\sin\phi\\
      -g\cos\theta\cos\phi
    \end{array}
\right]
\end{split}
\end{align}\

\noindent
式(115)をみると$\psi$成分が消えていることがわかる.つまり加速度センサだけではヨー角を補正することができない.また地球の地磁気ベクトルを$m_{r}=[m_{n},0,m_{c}]^{T}$とした時,キャリブレーションされた磁気センサから得られる3次元磁気データ$[m_{x},m_{y},m_{z}]^{T}$と地磁気ベクトルの関係は次のようになる.

\begin{align}
\begin{split}
\left[
    \begin{array}{c}
      m_{x} \\
      m_{y} \\
      m_{z}
    \end{array}
\right]
&=
\left[
    \begin{array}{ccc}
      \cos\theta\cos\psi & \cos\theta\sin\psi & -\sin\theta\\
      \sin\phi\sin\theta\cos\psi & \sin\phi\sin\theta\sin\psi+\cos\phi\cos\psi & \sin\phi\cos\theta \\
      \cos\phi\sin\theta\cos\psi & \cos\phi\sin\theta\sin\psi-\sin\phi\cos\psi & \cos\phi\cos\theta
    \end{array}
\right]
\left[
    \begin{array}{c}
      m_{n}\\
      0 \\
      m_{c}
    \end{array}
\right]\\
&=
\left[
    \begin{array}{c}
      m_{x}\cos\theta\cos\psi-m_{z}\sin\theta\\
      m_{x}\sin\phi\sin\theta\cos\psi+m_{z}\cos\theta\sin\phi\\
      m_{x}\cos\phi\sin\theta\cos\psi+m_{z}\cos\theta\cos\phi
    \end{array}
\right]
\end{split}
\end{align}\

\noindent
したがって$\textbf{y}(t)=[a_{x},a_{y},a_{z},m_{x},m_{y},m_{z}]^{T}$とおき式(115)と式(116)をまとめると次式のようになる.

\begin{align}
\textbf{y}(t)=\textbf{h}_{t}(\textbf{x}(t))
\end{align}\

\noindent
ただし,$\textbf{h}_{t}(\cdot)$

\begin{align}
\textbf{h}_{t}(\textbf{x}(t))=
\left[
    \begin{array}{c}
      g\sin\theta\\
      -g\cos\theta\sin\phi\\
      -g\cos\theta\cos\phi\\
      m_{x}\cos\theta\cos\psi-m_{z}\sin\theta\\
      m_{x}\sin\phi\sin\theta\cos\psi+m_{z}\cos\theta\sin\phi\\
      m_{x}\cos\phi\sin\theta\cos\psi+m_{z}\cos\theta\cos\phi
    \end{array}
\right]
\end{align}\

\noindent
式(118)に加速度センサと地磁気センサから得られるノイズ$[\delta a_{x},\delta a_{y},\delta a_{z}]^{T},[\delta m_{x},\delta m_{y},\delta m_{z}]^{T}$を考慮すると観測方程式は.

\begin{align}
\textbf{y}(t)=\textbf{h}_{t}(\textbf{x}(t))+\textbf{w}
\end{align}\

\noindent
ただし$\textbf{w}$は

\begin{align}
\textbf{w}=
[\delta a_{x},\delta a_{y},\delta a_{z},\delta m_{x},\delta m_{y},\delta m_{z}]^{T}
\end{align}\\

以上より,式(113)の状態方程式と式(119)の観測方程式より構成されるシステムの離散時間プロセルモデルが導出された.このプロセスモデルに対して拡張カルマンフィルタを適用することで姿勢推定アルゴリズムを得ることができる.


\subsection{姿勢制御}
クアッドロータの姿勢制御に関しては,様々な制御法が研究されているが,その中でも運動モデルをもとに制御器を設計でき,制御のための計算量も少ないIntegral Backstepping制御に着目した\cite{o}\cite{q}.

\subsubsection{Integral Backstepping制御}

まずBackstepping制御について説明する.次のように表されるシステムを考える.

\begin{align}
\left\{
\begin{aligned}
\dot{x_{1}} & = x_{2} \\
\dot{x_{2}} & = au+b
\end{aligned}
\right.
\end{align}\

\noindent
ここで$x_{1},x_{2}$は状態変数で,$u$は入力,$a,b$は定数である.まず変数変換を行い

\begin{align}
e_{1}=x_{1d}-x_{1},e_{2}=c_{1}e_{1}+\dot{x}_{1d}-x_{2}
\end{align}\

\noindent
とおく.ただし$x_{1d}$は$x_{1}$の目標値であり,$c_{1}$は正の定数である.そしてリアプノフ関数$V$を次式のように設定する.

\begin{align}
V(e_{1},e_{2})=\frac{1}{2}e_{1}^{2}+\frac{1}{2}e_{2}^{2}
\end{align}\

\noindent
式(123)を時間微分すると

\begin{align}
\dot{V}(e_{1},e_{2})=e_{1}\dot{e}_{1}+e_{2}\dot{e}_{2}
\end{align}\

\noindent
となる.$e_{1}$と$e_{2}$の時間微分はそれぞれ

\begin{align}
\dot{e}_{1}=\dot{x}_{1d}-x_{2}=\dot{x}_{1d}-(c_{1}e_{1}+\dot{x}_{1d}-e_{2})=-c_{1}e_{1}+e_{2}
\end{align}
\begin{align}
\dot{e}_{2}=c_{1}\dot{e}_{1}+\ddot{x}_{1d}-\dot{x}_{2}=c_{1}(-c_{1}e_{1}+e_{2})+\ddot{x}_{1d}-(au+b)
\end{align}\

\noindent
となるため,式(125)と式(126)を式(124)に代入すると

\begin{align}
\dot{V}(e_{1},e_{2})=-c_{1}e_{1}^{2}+e_{1}e_{2}-c_{1}^{2}e_{1}e_{2}+c_{1}e_{2}^{2}+\ddot{x}_{1d}e_{2}-(au+b)e_{2}
\end{align}\

\noindent
となる.ここで制御量$u$を次式のように設定する.

\begin{align}
u=\frac{1}{a}((1-c_{1}^{2})e_{1}+(c_{1}+c_{2})e_{2}+\ddot{x}_{1d}-b)
\end{align}\

\noindent
ただし,$c_{2}$は正の定数である.式(124)は

\begin{align}
\dot{V}(e_{1},e_{2})=-c_{1}e_{1}^{2}-c_{2}e_{2}^{2}\leq 0
\end{align}\

\noindent
となり,式(121)に示すシステムは安定となる.したがって制御量$u$を式(128)のように設定することで$e_{1}$は0に収束し,$x_{1}$は目標値$x_{1d}$に収束する.式(128)に式(121)を代入し変形すると次式のようになる.

\begin{align}
u=\frac{1}{a}((1+c_{1}c_{2})(x_{1d}-x_{1})+(c_{1}+c_{2})(\dot{x}_{1d}-\dot{x}_{1})+\ddot{x}_{1d}-b)
\end{align}\

\noindent
式(130)の最初の2項をみるとこれは明らかにPD制御である.また3,4項目をみると式(121)のダイナミクスが含まれていることがわかる.

次にIntegral Backstepping制御について説明する.Integral Backstepping制御の場合

\begin{align}
e_{1}=x_{1d}-x_{1},e_{2}=c_{1}e_{1}+\dot{x}_{1d}+\lambda p-x_{2}
\end{align}\

\noindent
とおく.ただし$\lambda$は正の定数であり,$p=\int_0^t e_{1}(\tau)d\tau$である.またリアプノフ関数$V$を次式のように設定する.

\begin{align}
V(e_{1},e_{2})=\frac{1}{2}\lambda p^{2}+\frac{1}{2}e_{1}^{2}+\frac{1}{2}e_{2}^{2}
\end{align}\

\noindent
また式(132)を時間微分すると

\begin{align}
\dot{V}(e_{1},e_{2})=\lambda p e_{1}+e_{1}\dot{e}_{1}+e_{2}\dot{e}_{2}
\end{align}\

\noindent
となる.$e_{1}$と$e_{2}$の時間微分はそれぞれ

\begin{align}
\dot{e}_{1}=\dot{x}_{1d}-x_{2}=\dot{x}_{1d}-(c_{1}e_{1}+\dot{x}_{1d}+\lambda p-e_{2})=-c_{1}e_{1}-\lambda p+e_{2}
\end{align}
\begin{align}
\dot{e}_{2}=c_{1}\dot{e}_{1}+\ddot{x}_{1d}+\lambda e_{1}-\dot{x}_{2}=c_{1}(-c_{1}e_{1}-\lambda p+e_{2})+\ddot{x}_{1d}+\lambda e_{1}-(au+b)
\end{align}\

\noindent
となる.ここで制御量$u$を次式のように設定する.

\begin{align}
u=\frac{1}{a}((1-c_{1}^{2}-\lambda)e_{1}+(c_{1}+c_{2})e_{2}+\lambda pc_{1}+\ddot{x}_{1d}-b)
\end{align}\

\noindent
ただし,$c_{2}$は正の定数である.したがって式(133)は

\begin{align}
\dot{V}(e_{1},e_{2})=-c_{1}e_{1}^{2}-c_{2}e_{2}^{2}\leq 0
\end{align}\

\noindent
となり,式(121)に示すシステムは安定になる.式(136)に式(131)を代入し変形すると次式のようになる.

\begin{align}
u=\frac{1}{a}((1+c_{1}c_{2}+\lambda)(x_{1d}-x_{1})+(c_{1}+c_{2})(\dot{x}_{1d}-\dot{x}_{1})+\lambda c_{2}\int_0^t (x_{1d}-x_{1})d\tau +\ddot{x}_{1d}-b)
\end{align}\

\noindent
式(138)の最初の3項をみるとこれは明らかにPID制御である.また4,5項目をみるとBackstepping制御と同様に式(121)のダイナミクスが含まれていることがわかる.


\subsubsection{姿勢制御アルゴリズム}

クアッドロータの姿勢の運動方程式は式(17)より

\begin{align} 
\left[
    \begin{array}{c}
      \dot{\phi} \\
      \dot{\theta} \\
      \dot{\psi}
    \end{array}
\right]
=
\left[
    \begin{array}{c}
      \frac{J_{2}-J_{3}}{J_{1}}\theta \psi \\
      \frac{J_{3}-J_{1}}{J_{2}}\psi \phi \\
      \frac{J_{1}-J_{2}}{J_{3}}\phi \theta 
    \end{array}
\right]
+
\left[
    \begin{array}{c}
      \frac{1}{J_{1}}\tau_{\phi} \\
      \frac{1}{J_{2}}\tau_{\theta} \\
      \frac{1}{J_{3}}\tau_{\psi} 
    \end{array}
\right]
\end{align}\

\noindent
ここで$[x_{1},x_{2},x_{3},x_{4},x_{5},x_{6}]^{T}=[\phi,\theta,\psi,\dot{\phi},\dot{\theta},\dot{\psi}]^{T}$とおき,次式のように書き換える.

\begin{align}
\left\{
\begin{aligned}
\dot{x}_{1} &= x_{4}\\
\dot{x}_{4} &= a_{1}x_{2}x_{3}+b_{1}\tau_{\phi}\\
\dot{x}_{2} &= x_{5}\\
\dot{x}_{5} &= a_{2}x_{3}x_{1}+b_{2}\tau_{\theta}\\
\dot{x}_{3} &= x_{6}\\
\dot{x}_{6} &= a_{3}x_{1}x_{2}+b_{3}\tau_{\psi}
\end{aligned}
\right.
\end{align}\

\noindent
ただし

\begin{align}
a_{1}=\frac{J_{2}-J_{3}}{J_{1}},a_{2}=\frac{J_{3}-J_{1}}{J_{2}},a_{3}=\frac{J_{1}-J_{2}}{J_{3}},b_{1}=\frac{1}{J_{1}}\,b_{2}=\frac{1}{J_{2}},b_{3}=\frac{1}{J_{3}}
\end{align}\

\noindent
まず$\phi$に関するIntegral Backstepping制御器を導出していく.最初に次式のように変数変換を行う.

\begin{align}
e_{1}=\phi_{d}-x_{1},e_{2}=c_{1}e_{1}+\dot{\phi}_{d}-x_{4}+\lambda_{1} p -x_{4}
\end{align}\

\noindent
ただし$\phi_{d}$は$\phi$の目標値であり,$c_{1},\lambda_{1}$は正の定数,$p=\int_0^t e_{1}(\tau)d\tau$である.またリアプノフ関数を次式のように設定する.

\begin{align}
V(e_{1},e_{2})=\frac{1}{2}\lambda_{1} p^{2}+\frac{1}{2}e_{1}^{2}+\frac{1}{2}e_{2}^{2}
\end{align}

\noindent
また式(143)を時間微分すると

\begin{align}
\dot{V}(e_{1},e_{2})=e_{1}\dot{e}_{1}+e_{2}\dot{e}_{2}
\end{align}\

\noindent
となる.$e_{1}$と$e_{2}$の時間微分はそれぞれ

\begin{align}
\dot{e}_{1}=\dot{\phi}_{d}-x_{2}=\dot{\phi}_{d}-(c_{1}e_{1}+\dot{\phi}_{d}+\lambda_{1} p-e_{2})=-c_{1}e_{1}-\lambda_{1} p+e_{2}
\end{align}
\begin{align}
\dot{e}_{2}=c_{1}\dot{e}_{1}+\ddot{\phi}_{d}+\lambda_{1} e_{1}-\dot{x}_{2}=c_{1}(-c_{1}e_{1}-\lambda_{1} p+e_{2})+\ddot{\phi}_{d}+\lambda_{1} e_{1}-a_{1}x_{2}x_{3}-b_{1}\tau_{\phi}
\end{align}\

\noindent
となる.ここでトルク$\tau_{\phi}$を次式のように設定する.

\begin{align}
\tau_{\phi}=\frac{1}{a}((1-c_{1}^{2}-\lambda_{1})e_{1}+(c_{1}+c_{2})e_{2}+\lambda_{1} pc_{1}+\ddot{\phi}_{d}-a_{1}x_{2}x_{3})
\end{align}\

\noindent
ただし,$c_{2}$は正の定数である.したがって$V(e_{1},e_{2})\leq 0$となりシステムは安定になる.式(147)に式(142)を代入して$\phi,\theta,\psi$で表すと

\begin{align}
\tau_{\phi}=\frac{1}{b_{1}}((1+c_{1}c_{2}+\lambda_{1})(\phi_{d}-\phi)+(c_{1}+c_{2})(\dot{\phi}_{d}-\dot{\phi})+c_{2}\lambda_{1}\int_0^t (\phi_{d}-\phi)d\tau-a_{1}x_{2}x_{3})
\end{align}\

\noindent
となる.ただし$\ddot{\phi}_{d}=0$とおいた.$\theta,\psi$に関しても同様に計算すると

\begin{align}
\tau_{\theta}=\frac{1}{b_{2}}((1+c_{3}c_{4}+\lambda_{2})(\theta_{d}-\theta)+(c_{3}+c_{4})(\dot{\theta}_{d}-\dot{\theta})+c_{4}\lambda_{2}\int_0^t (\theta_{d}-\theta)d\tau-a_{2}x_{3}x_{1})
\end{align}

\begin{align}
\tau_{\psi}=\frac{1}{b_{3}}((1+c_{5}c_{6}+\lambda_{3})(\psi_{d}-\psi)+(c_{5}+c_{6})(\dot{\psi}_{d}-\dot{\psi})+c_{4}\lambda_{3}\int_0^t (\psi_{d}-\psi)d\tau-a_{3}x_{1}x_{2})
\end{align}\

\noindent
ここで,$c_{3},c_{4},c_{5},c_{6},\lambda_{2},\lambda_{3}$は正の定数であり,$\theta_{d},\psi_{d}$はそれぞれ$\theta,\psi$の目標値である.以上で$\phi,\theta,\psi$に関する制御器をもとめることができた.

\newpage
\section{姿勢推定アルゴリズムの評価}
\subsection{地磁気センサのキャリブレーション}
周囲の環境が磁気センサに及ぼす影響を調べるため,京都工芸繊維大学の10号館前,食堂前,食堂の階段下の3箇所において3軸の地磁気センサを用いて地磁気を計測した.(Fig. 45).Fig. 45のオレンジ色の丸で囲われた場所が計測地点である.10号館前は大きな磁性体が周りにない環境であり,食堂前は,食堂が鉄骨の建物であるため,大きな磁性体が傍にある環境である.食堂の階段下は,階段が鋼鉄製であるため,上を磁性体に囲われた環境である.


\begin{figure}[H]
\begin{center}
\includegraphics[height=90mm]{./Fig/Location_of_magnetometer_calibration_experiment.png}
\caption{Location of magnetometer calibration experiment.}
\label{Location_of_magnetometer_calibration_experiment}
\end{center}
\end{figure}

\begin{figure}[H]
\begin{center}
\includegraphics[height=50mm]{./Fig/Experimental_equipment_for_geomagnetic_measurement.png}
\caption{Experimental equipment for geomagnetic measurement.}
\label{Experimental_equipment_for_geomagnetic_measurement}
\end{center}
\end{figure}

計測に使用した実験装置をFig. 46に示す.この装置にはWi-Fiモジュールが搭載されており,センサから得られたデータをWi-Fi経由で送信することができる.今回,地磁気の計測装置をあらゆる方向に回転させ50 Hzで3次元の地磁気を計測した.それぞれの場所における計測結果をFigs. 47~49に示す.

\begin{figure}[H]
\begin{center}
\includegraphics[height=50mm]{./Fig/Raw_data_from_the_magnetometer_in_front_of_Building_No10.png}
\caption{Raw data from the magnetometer in front of Building No.10.}
\label{Raw_data_from_the_magnetometer_in_front_of_Building_No10}
\end{center}
\end{figure}

\begin{figure}[H]
\begin{center}
\includegraphics[height=50mm]{./Fig/Raw_data_from_the_magnetometer_in_front_of_the_cafeteria.png}
\caption{Raw data from the magnetometer in front of the cafeteria.}
\label{Raw_data_from_the_magnetometer_in_front_of_the_cafeteria}
\end{center}
\end{figure}

\begin{figure}[H]
\begin{center}
\includegraphics[height=50mm]{./Fig/Raw_data_from_the_magnetometer_under_the_stairs_of_the_cafeteria.png}
\caption{Raw data from the magnetometer under the stairs of the cafeteria.}
\label{Raw_data_from_the_magnetometer_under_the_stairs_of_the_cafeteria}
\end{center}
\end{figure}

各場所の計測結果を見ると,測定値の分布の中心と原点がずれていることがわかる.また,それぞれの測定値の分布は円ではなく楕円を描いている.このことから,地磁気センサから得られた値をそのまま姿勢角推定に使用した場合,正しい姿勢角が得られないということがわかる.さらに,Fig. 49の食堂の階段下で計測した結果を見ると10号館前や食堂前での値に比べ乱れていることがわかる.これは近くに鋼鉄製の階段があり,それにより地磁気が乱されていることが影響していると考えられる.
 

次に10号館前で計測した値を楕円体に最小二乗法を用いてフィッティングさせ,$\textbf{A}_{h}^{-1},\textbf{b}$をそれぞれ求めた.$\textbf{A}_{h}^{-1},\textbf{b}$の具体的な値を次式に示す.

\begin{align} 
\textbf{A}_{h}^{-1}=
\left[
    \begin{array}{ccc}
      2.049\times10^{-2}&2.617\times10^{-4}&-4.073\times10^{-4}\\
      2.617\times10^{-4}&2.151\times10^{-2}&-1.803\times10^{-5} \\
      -4.073\times10^{-4}&-1.803\times10^{-5}&2.094\times10^{-2} 
    \end{array}
\right],
\textbf{b}=
\left[
    \begin{array}{c}
      15.472 \\
      1.374 \\
      -35.567
    \end{array}
\right]
\end{align}\

\noindent
そして求めた$\textbf{A}_{h}^{-1},\textbf{b}$を式(75)に代入し,その式を用いてそれぞれの地点で計測したデータを変換した.変換した結果をFigs. 50~52変換の際は後で姿勢角を算出するために原点中心かつ半径1の球に変換している.

\begin{figure}[H]
\begin{center}
\includegraphics[height=50mm]{./Fig/Calibrated_geomagnetic_data_in_front_of_Building_No10.png}
\caption{Calibrated geomagnetic data in front of Building No.10.}
\label{Calibrated_geomagnetic_data_in_front_of_Building_No10}
\end{center}
\end{figure}

\begin{figure}[H]
\begin{center}
\includegraphics[height=50mm]{./Fig/Calibrated_geomagnetic_data_in_front_of_the_cafeteria.png}
\caption{Calibrated geomagnetic data in front of the cafeteria.}
\label{Calibrated_geomagnetic_data_in_front_of_the_cafeteria}
\end{center}
\end{figure}

\begin{figure}[H]
\begin{center}
\includegraphics[height=50mm]{./Fig/Calibrated_geomagnetic_data_under_the_stairs_of_the_cafeteria.png}
\caption{Calibrated geomagnetic data under the stairs of the cafeteria.}
\label{Calibrated_geomagnetic_data_under_the_stairs_of_the_cafeteria}
\end{center}
\end{figure}

Fig. 50の変換後の分布をみるとおおよそ原点中心かつ半径1の球に変換されていることがわかる.次にFig. 51を見ると原点中心ではあるが,半径が1より大きくなっていることがわかる.これは10号館前と食堂前で周囲の環境が変化し,地磁気の強度に影響を及ぼしたためと考えられる.最後にFig. 52を見ると形は歪であるが,原点中心の円になっていることがわかる.ただし,半径は1より小さくなっている.これもFig. 51同様,周囲環境が変化したためと考えられる.

以上から10号館前で計測した値を用いて変換を行った結果,半径は若干変化するものの原点中心の円に変換することができた.したがって,最初に地磁気のキャリブレーションを行えば,ある程度周囲の環境が変化したとしても,キャリブレーション後の値は姿勢角推定に用いることができるということがわかった.

\subsection{ロール角及びピッチ角推定の評価}
4.1.3で述べた拡張カルマンフィルタを用いた姿勢角推定アルゴリズムの評価を行った.評価方法としては,姿勢角を変化させ,ポテンショメータから得られる角度を真値として推定した角度との比較を行った.

\begin{figure}[H]
\begin{center}
\includegraphics[height=60mm]{./Fig/Schematic_diagram_of_experimental_apparatus_for_attitude_angle_estimation.png}
\caption{Schematic diagram of experimental apparatus for attitude angle estimation.}
\label{Schematic_diagram_of_experimental_apparatus_for_attitude_angle_estimation}
\end{center}
\end{figure}

Fig. 53に実験装置の概略図を示す.この実験装置を用いてピッチ角とロール角の評価を行.ポテンショメータで角度が計測できるアームの上に9軸のIMUとシングルボードコンピュータがのっている.このコンピュータは姿勢角推定アルゴリズムの計算とポテンショメータの角度計測を同時に行い,得られた計測データをPCにWi-Fi経由で送信することができる.アームは0 deg, 10 deg, 20 degと段階的に手動で回転させまた20 deg, 10 deg, 0 degと戻す.これをピッチ角とロール角に関して行う.今回,角度の計測は100 Hzで行った.

\begin{figure}[H]
\begin{center}
\includegraphics[height=80mm]{./Fig/Overview_of_the_experimental_apparatus_for_pitch_angle_estimation.png}
\caption{Overview of the experimental apparatus for pitch angle estimation.}
\label{Overview_of_the_experimental_apparatus_for_pitch_angle_estimation}
\end{center}
\end{figure}

最初にピッチ角の評価を行う.Fig. 54にピッチ角の計測に使用した実験装置を示す.アームはピッチ角の方向のみに回転することができるようになっている.Fig. 55にポテンショメータから得られたピッチ角度と推定したピッチ角の時系列データを示す.またFig. 56に真値との偏差を示す.偏差は次式で定義する.

\begin{align}
Error =Estimated\ angle-True\ angle
\end{align}\

\begin{figure}[H]
\begin{center}
\includegraphics[height=70mm]{./Fig/Estimated_pitch_angle_and_true_value_in_pitch_angle_estimation.png}
\caption{Estimated pitch angle and true value in pitch angle estimation.}
\label{Estimated_pitch_angle_and_true_value_in_pitch_angle_estimation}
\end{center}
\end{figure}

\begin{figure}[H]
\begin{center}
\includegraphics[height=70mm]{./Fig/Error_in_pitch_angle_estimation.png}
\caption{Error in pitch angle estimation.}
\label{Error_in_pitch_angle_estimation}
\end{center}
\end{figure}

\noindent
Fig. 55においてポテンショメータから得られた角度が細かく変化しているのは電気的なノイズによるものである.Fig. 56の定常状態における偏差を見ると,おおよそ$\pm$1 deg以内でピッチ角を推定できていることがわかる.また重力加速度を用いてピッチ角の補正を行っているため,推定値のドリフトも発生していないことがわかる.

次にロール角の評価を行う.Fig. 57にロール角の計測に使用した実験装置を示す.アームはロール角の方向のみに回転することができるようになっている.Fig. 58にポテンショメータから得られたロール角と推定したロール角の時系列データを示す.またFig. 59に真値との偏差を示す.

\begin{figure}[H]
\begin{center}
\includegraphics[height=80mm]{./Fig/Overview_of_the_experimental_apparatus_for_roll_angle_estimation.png}
\caption{Overview of the experimental apparatus for roll angle estimation.}
\label{Overview_of_the_experimental_apparatus_for_roll_angle_estimation}
\end{center}
\end{figure}

\begin{figure}[H]
\begin{center}
\includegraphics[height=70mm]{./Fig/Estimated_roll_angle_and_true_value_in_roll_angle_estimation.png}
\caption{Estimated roll angle and true value in roll angle estimation.}
\label{Estimated_roll_angle_and_true_value_in_roll_angle_estimation}
\end{center}
\end{figure}

\begin{figure}[H]
\begin{center}
\includegraphics[height=70mm]{./Fig/Error_in_roll_angle_estimation.png}
\caption{Error in roll angle estimation.}
\label{Error_in_roll_angle_estimation}
\end{center}
\end{figure}

\noindent
Fig. 59の定常状態における偏差を見ると,ピッチ角と同様ロール角もおおよそ$\pm$1 deg以内で推定できていることがわかる.また重力加速度を用いてロール角の補正を行っているため,推定値のドリフトも発生していないことがわかる.

\subsection{ヨー角推定の評価}

\begin{figure}[H]
\begin{center}
\includegraphics[height=80mm]{./Fig/Overview_of_the_experimental_apparatus_for_yaw_angle_estimation.png}
\caption{Overview of the experimental apparatus for yaw angle estimation.}
\label{Overview_of_the_experimental_apparatus_for_yaw_angle_estimation}
\end{center}
\end{figure}

Fig. 60にヨー角の計測に使用した実験装置を示す.実験装置にはターンテーブルがついており,ヨー角の方向のみに回転することができる.またその角度もポテンショメータにより計測することができる.ピッチ角,ロール角の評価と同じように姿勢角推定とポテンショメータの角度計測を同時に行い,得られた計測データをPCにWi-Fi経由で送信し記録した.計測中,ターンテーブルは0 deg, 60 deg, 120 degと段階的に手動で回転させまた120 deg, 60 deg, 0 degと戻し,今度は0 deg, -60 deg, -120 degと段階的に回転させ-120 deg, -60 deg, 0 degのように戻した.角度の計測は100 Hzで行った.Fig. 61にポテンショメータから得られたヨー角と推定したヨー角の時系列データを示す.またFig. 62に真値との偏差を示す.

\begin{figure}[H]
\begin{center}
\includegraphics[height=70mm]{./Fig/Estimated_yaw_angle_and_true_value_in_yaw_angle_estimation.png}
\caption{Estimated yaw angle and true value in yaw angle estimation.}
\label{Estimated_yaw_angle_and_true_value_in_yaw_angle_estimation}
\end{center}
\end{figure}

\begin{figure}[H]
\begin{center}
\includegraphics[height=70mm]{./Fig/Error_in_yaw_angle_estimation.png}
\caption{Error in yaw angle estimation.}
\label{Error_in_yaw_angle_estimation}
\end{center}
\end{figure}

\noindent
Fig. 62の定常状態における偏差を見ると,最大で2 degのヨー角の推定誤差があることがわかった.また地磁気を用いてヨー角の補正を行っているため,推定値のドリフトも発生していないことがわかる.

ヨー角の推定は地磁気を用いて補正しているため,周囲環境が変化し,周りの磁場が変化すると推定値に影響を及ぼす.アッドロータのロータが回転すれば電流がモータに流れるため,センサ周りの磁場が変化するため,その影響を調べる必要がある.そこでロータが回転している状態でヨー角推定の評価を行った.計測はロータをおおよそ12000 rpm程度で回転させながら5.3と同様にターンテーブルを回転させて行った.回転数を12000 rpmに設定した理由は,12000 rpmで回転させるとクアッドロータの浮上時の推力が得られるからである.これにより飛行中に角度推定を行っている状況を再現することができる.角度の計測は100 Hzで行った.実験装置をFig. 63に示す.

\begin{figure}[H]
\begin{center}
\includegraphics[height=80mm]{./Fig/Overview_of_the_experimental_apparatus_for_yaw_angle_estimation_with_rotors.png}
\caption{Overview of the experimental apparatus for yaw angle estimation with rotors.}
\label{Overview_of_the_experimental_apparatus_for_yaw_angle_estimation_with_rotors}
\end{center}
\end{figure}

\noindent
Fig. 64にポテンショメータから得られたヨー角と推定したヨー角の時系列データを示す.またFig. 65に真値との偏差を示す.

\begin{figure}[H]
\begin{center}
\includegraphics[height=70mm]{./Fig/Estimated_roll_angle_and_true_value_with_rotors.png}
\caption{Estimated yaw angle and true value with rotors.}
\label{Estimated_roll_angle_and_true_value_with_rotors}
\end{center}
\end{figure}

\begin{figure}[H]
\begin{center}
\includegraphics[height=70mm]{./Fig/Error_in_yaw_angle_estimation_with_rotors.png}
\caption{Error in yaw angle estimation with rotors.}
\label{Error_in_yaw_angle_estimation_with_rotors}
\end{center}
\end{figure}

\noindent
Fig. 65の定常状態における偏差を見ると,最大で4 deg程度のヨー角の推定誤差があり,Fig. 62と比較すると偏差が大きくなっていることがわかった.これはモータが回転することでセンサ周りの磁場が変化し,ヨー角推定に影響を及ぼしているからである.




\newpage
\section{姿勢制御アルゴリズムの評価}
\subsection{ロール角及びピッチ角制御の評価}

\begin{figure}[H]
\begin{center}
\includegraphics[height=80mm]{./Fig/Overview_of_the_experimental_apparatus_for_pitch_angle_control.png}
\caption{Overview of the experimental apparatus for pitch angle control.}
\label{Overview_of_the_experimental_apparatus_for_pitch_angle_control}
\end{center}
\end{figure}

Integral Backstepping制御によるピッチ角の制御性能を単独で評価するために,Fig. 66のような実験装置を用意した.この実験装置は,クアッドロータがピッチ角のみ回転できるようになっている.実験では指令角をステップ状に0 deg, 20 deg, 0 deg, -20 deg, 0 degの順番で入力し,姿勢制御とその時の姿勢角を記録した.姿勢制御における周波数は100 Hzである.
Fig. 67にピッチ角の指令値と制御中のピッチ角の時系列データを示す.

\begin{figure}[H]
\begin{center}
\includegraphics[height=70mm]{./Fig/Pitch_angle_and_reference_in_pitch_angle_control_experiment.png}
\caption{Pitch angle and reference in pitch angle control experiment.}
\label{Pitch_angle_and_reference_in_pitch_angle_control_experiment}
\end{center}
\end{figure}

\noindent
Fig. 67を見ると指令値に対して追従するように姿勢制御ができていることがわかる.また目標値に到達するまでの時間はおよそ0.4 secであることがわかった.

ピッチ角制御の評価と同様にIntegral Backstepping制御によるロール角の制御性能も単独で評価した.実験装置をFig. 68に示す.この実験装置は,クアッドロータがロール角のみ回転できるようになっている.実験では指令角をステップ状に0 deg, 20 deg, 0 deg, -20 deg, 0 degの順番で入力し,姿勢制御とその時の姿勢角を記録した.姿勢制御における周波数は100 Hzである.

\begin{figure}[H]
\begin{center}
\includegraphics[height=80mm]{./Fig/Overview_of_the_experimental_apparatus_for_roll_angle_control.png}
\caption{Overview of the experimental apparatus for roll angle control.}
\label{Overview_of_the_experimental_apparatus_for_roll_angle_control}
\end{center}
\end{figure}

Fig. 69にロール角の指令値と制御中のロール角の時系列データを示す.

\begin{figure}[H]
\begin{center}
\includegraphics[height=70mm]{./Fig/Roll_angle_and_reference_in_roll_angle_control_experiment.png}
\caption{Roll angle and reference in roll angle control experiment.}
\label{Roll_angle_and_reference_in_roll_angle_control_experiment}
\end{center}
\end{figure}

\noindent
Fig. 69を見るとピッチ角と同じように指令値に対して追従するように姿勢制御ができていることがわかる.また目標値に到達するまでの時間はおよそ0.4 secであることがわかった.


\subsection{ヨー角制御の評価}

\begin{figure}[H]
\begin{center}
\includegraphics[height=80mm]{./Fig/Overview_of_the_experimental_apparatus_for_yaw_angle_control.png}
\caption{Overview of the experimental apparatus for yaw angle control.}
\label{Overview_of_the_experimental_apparatus_for_yaw_angle_control}
\end{center}
\end{figure}

Integral Backstepping制御によるヨー角の制御性能を単独で評価するために,Fig. 70のような実験装置を用意した.クアッドロータがターンテーブルの上に固定されており,ヨー角のみ回転できるようになっている.実験では指令角をステップ状に0 deg, 60 deg, 0 deg, -60 deg, 0 degの順番で入力し,姿勢制御とその時の姿勢角を記録した.姿勢制御における周波数は100Hzである.
Fig. 71にヨー角の指令値と制御中のヨー角の時系列データを示す.

\begin{figure}[H]
\begin{center}
\includegraphics[height=70mm]{./Fig/Yaw_angle_and_reference_in_yaw_angle_control_experiment.png}
\caption{Yaw angle and reference in yaw angle control experiment.}
\label{Yaw_angle_and_reference_in_yaw_angle_control_experiment}
\end{center}
\end{figure}

\noindent
Fig. 71を見ると指令値に対して追従するように姿勢制御ができていることがわかるが,目標値に到達するまでの時間はおよそ1.0 secであり,ロール角,ピッチ角の制御と比較して遅い.これはヨー角の制御はモーターの反トルクのみで行うため,推力により発生するピッチ,ロール方向のトルクに比べ小さいからである.しかし,今回ヨー角が激しく動くような動きは想定していないため,ヨー角の応答時間に関しては特に問題はない.


\subsection{ホバリング実験}
ピッチ角,ロール角,ヨー角を組み合わせた制御性能を評価するため,クアッドロータのホバリング実験を行った.ホバリング実験では姿勢角の指令値を0 degに設定し,スロットルのみ操縦者がリモコンを介して操作する.またその時のクアッドロータの姿勢角と位置を記録した.位置はクアッドロータにマーカーを付け,ステレオカメラで計測している.

Fig. 72とFig. 73ににホバリング中のクアッドロータの姿勢角と位置を示す.

\begin{figure}[H]
\begin{center}
\includegraphics[height=70mm]{./Fig/Attitude_angle_of_the_quad-rotor_in_hovering_experiment.png}
\caption{Attitude angle of the quad-rotor in hovering experiment.}
\label{Attitude_angle_of_the_quad-rotor_in hovering_experiment}
\end{center}
\end{figure}

\begin{figure}[H]
\begin{center}
\includegraphics[height=70mm]{./Fig/Position_of_the_quad-rotor_in_hovering_experiment.png}
\caption{Position of the quad-rotor in hovering experiment.}
\label{Position_of_the_quad-rotor_in hovering_experiment}
\end{center}
\end{figure}

\noindent
Fig. 72とFig. 73を見ると浮き上がる直前や機体が地面に近い時(0$\sim$10 sec)は姿勢角及び位置が大きく変化していることがわかる.これは地面に近いため,ロータからの風が機体に影響与えているからである.十分に地面から離れると(0$\sim$10 sec)姿勢角及び位置の変化は緩やかになっていくことがわかる.ピッチ角とロール角に関しては,指令値から$\pm$2 degのところで安定し,ヨー角に関しては指令値から$\pm$1 degのところで安定していることがわかる.



\newpage
%%%%%%%%%%%%%%%%%%%%%%%%%%%%%%%%%%%%%%%%%%%%%%%%%%%%%%%%%%%%%%%%%%%%%%%%%%%%%%%%%%%%%%%
%%%%%%%%%%%%%%%%%%%%%% 位置制御システム %%%%%%%%%%%%%%%
%%%%%%%%%%%%%%%%%%%%%%%%%%%%%%%%%%%%%%%%%%%%%%%%%%%%%%%%%%%%%%%%%%%%%%%%%%%%%%%%%%%%%%%%%%

\section{位置制御システム}

\begin{figure}[H]
\begin{center}
\includegraphics[height=60mm]{./Fig/Block_diagram_of_the_position_control_system.png}
\caption{Block diagram of the position control system.}
\label{Schematic_representation_of_each_instrumentation_error}
\end{center}
\end{figure}

位置制御システムのブロック線図をFig. 74に示す.クアッドロータは位置制御と姿勢制御が互いに関係し影響を及ぼす.そこで姿勢制御と位置制御を分離して考える逆ダイナミクス法を用いた位置制御システムを構築する\cite{r}.位置制御系は姿勢制御計に目標指令を与え,姿勢制御系は与えられた目標指令に機体を追従させる.位置や姿勢の推定には拡張カルマンフィルタを用いている.ここではまずIMU,UWB,距離センサを用いた位置推定法について述べ,その後Integral Backstepping制御を用いた位置制御システムについて述べる.


\subsection{位置推定}
\subsubsection{UWBによる測距}

\begin{figure}[H]
\begin{center}
\includegraphics[height=60mm]{./Fig/Comparison_of_spectrum.png}
\caption{Comparison of spectrum.}
\label{Comparison_of_spectrum}
\end{center}
\end{figure}

超広帯域無線通信(Ultra Wide Band, UWB)とは非常に広い帯域幅にわたって電力を拡散させて高速通信を可能とする無線システムである.Fig. 75にUWBと既存の無線通信方式とのスペクトラムの比較を示す.使用する周波数帯域は3.1 GHzから10.6 GHzで帯域幅は7.5 GHzと従来の無線の数百kHzやWi-Fiの20 MHzに比べてかなり広い帯域を利用している.また送信出力は-41.3 dBm/MHz以下に制限されており、これは家庭用テレビやパソコン等の一般の電子機器等が発生する雑音レベルの約500分の1以下という小さな出力で通信を行う.またUWBはナノ秒オーダーのパルスを通信に用いるため,電波の到着時間を高精度に測定でき,測距・測位を高い精度で行うことができる.現在decaWave社からUWB対応の無線通信トランシーバモジュールDWM1000が販売されており,最大10 cmの測距精度と最大300 mの通信範囲が特徴である\cite{j}.
次にUWB通信を用いた測距原理について説明する.2つ以上のUWB通信モジュールを用意し,互いに通信を行う.この時UWBの信号の伝搬時間を測定し,その測定時間$ToF$と光の速度$c$を乗算することでモジュール間の距離$d$を求めることができる(式(153)).

\begin{align}
d = Tof \times c
\end{align}\

\begin{figure}[H]
\begin{center}
\includegraphics[height=60mm]{./Fig/Ranging_protocol_of_the_UWB_module.png}
\caption{Ranging protocol of the UWB module.}
\label{Ranging_protocol_of_the_UWB_module}
\end{center}
\end{figure}

\noindent
Fig. 76にUWBの通信プロトコルを示す.3次元の位置推定するためにはUWBモジュールはクアッドローター搭載用に1個,基地局用に3個以上必要となる.そのためお互いのモジュールを識別するためにまず基地局側からIDを送信する.その後クアッドローター側から応答があり,最後にIDと基地局側のタイムスタンプ($T_{SP},T_{RR},T_{SF}$)を送信する.この送られてきたタイムスタンプとクアッドロータ側のタイムスタンプ($T_{RP},T_{SR},T_{RF}$)を使って次式のように通信時間$ToF$を計算することができる.

\begin{align}
ToF = \frac{1}{4}((T_{SP}-T_{RR})-(T_{SR}-T_{RP})+(T_{RF}-T_{SR})-(T_{SF}-T_{RR}))
\end{align}\

\begin{figure}[H]
\begin{center}
\includegraphics[height=60mm]{./Fig/Positional_relationship_between_the_anchors_and_the_quad-rotor.png}
\caption{Positional relationship between the anchors and the quad-rotor.}
\label{Positional_relationship_between_the_anchors_and_the_quad-rotor}
\end{center}
\end{figure}

最後にUWBを使って三角測量によりクアッドロータの3次元位置を推定する方法について説明する.Fig. 77に基地局とクアッドロータの位置関係を示す.3つの基地局はあらかじめ橋梁等に設置されており,その三次元位置は既知であるとする(ここでは基地局が3箇所の場合について説明しているが,3箇所以上ある場合も同様に3次元位置を求めることができる.).各基地局とクアッドロータが通信を行い,お互いの距離$d_{1},d_{2},d_{3}$が測定できたとすると,その距離とクアッドロータの3次元座標$x^{r},y^{r},z^{r}$の関係は次式で表される.

\begin{align} 
\left[
    \begin{array}{c}
      d_{1} \\
      d_{2} \\
      d_{3}
    \end{array}
\right]
=
\left[
    \begin{array}{c}
      \sqrt{(x^{r}-x_{1})^{2}+(y^{r}-y_{1})^{2}+(z^{r}-z_{1})^{2}} \\
      \sqrt{(x^{r}-x_{2})^{2}+(y^{r}-y_{2})^{2}+(z^{r}-z_{2})^{2}} \\
      \sqrt{(x^{r}-x_{3})^{2}+(y^{r}-y_{3})^{2}+(z^{r}-z_{3})^{2}} 
    \end{array}
\right]
\end{align}\

\noindent
式(155)を$x^{r},y^{r},z^{r}$について解くことで3次元位置座標をもとめることができる.

また今回製作したUWB基地局をFig. 78に示す.この基地局にはクアッドロータと同様のDacaWave社製のDWM1000が搭載されている.Fig. 78からわかるようにとても小型であるため,基地局の設置は容易に行える.

\begin{figure}[H]
\begin{center}
\includegraphics[height=45mm]{./Fig/Anchor_unit.png}
\caption{Anchor unit.}
\label{Anchor_unit}
\end{center}
\end{figure}

\subsubsection{距離センサによる高度推定}
橋梁の下を飛行するためには橋梁との衝突を避けるために高度の推定が重要になる.そこで今回の位置推定システムに距離センサを追加することで高度の推定精度の向上を狙った.しかし距離センサで高度を推定する場合,Fig. 79のように姿勢角が変化した時,距離センサから得られる値$d_{h}$と高度に差が生じてしまう.

\begin{figure}[H]
\begin{center}
\includegraphics[height=60mm]{./Fig/Relationship_between_measurement_value_from_distance_sensor_and_altitude.png}
\caption{Relationship between measurement value from distance sensor and altitude.}
\label{Relationship_between_measurement_value_from_distance_sensor_and_altitude}
\end{center}
\end{figure}

\noindent
地面が水平であると仮定すると,距離センサからのデータを姿勢角を用いて補正すると次式のようになる.

\begin{align}
z^{r}= d_{h}\cos\theta\cos\phi
\end{align}\

\noindent
式(156)を用いることで距離センサから得られる値から高度を推定することができる.


\subsubsection{位置推定アルゴリズム}
ここでは拡張カルマンフィルタを用いた位置の推定アルゴリズムの導出を行う.はじめに状態方程式を求める.機体の速度ベクトル$v_{r}=[v_{x},v_{y},v_{z}]^{T}$と機体の位置ベクトル$p_{r}=[x_{r},y_{r},z_{r}]^{T}$の関係は次式で表される.

\begin{align}
\frac{d}{dt}p_{r}=v_{r}
\end{align}\

\noindent
速度ベクトルの時間微分$\dot{v}_{r}$は加速度センサから得られるデータ$a_{b}[a_{x},a_{y},a_{z}]^{T}$と回転行列$R$を用いて次のように表される.

\begin{align}
\frac{d}{dt}v_{r}=Ra_{b}-g_{r}
\end{align}\

\noindent
ここで$g_{r}=[0,0,-g]^{T}$である.またオイラー角$\Theta=[\phi,\theta,\psi]^{T}$の時間微分と機体角速度$\omega=[p,q,r]^{T}$の関係式は式(11)より

\begin{align}
\frac{d}{dt}\Theta = W\omega
\end{align}\

\noindent
となる.状態ベクトルを$\textbf{x}=[\Theta,p_{r},v_{r}]^{T}$とおき式(157)~(159)をまとめると次のようになる.

\begin{align}
\dot{\textbf{x}}(t)=\textbf{f}(\textbf{x}(t))
\end{align}\

\noindent
ただし$f(\cdot)$は

\begin{align} 
\textbf{f}(\textbf{x}(t))
=
\left[
    \begin{array}{c}
      W\omega \\
      v_{r} \\
      Ra_{b}-g_{r} 
    \end{array}
\right]
\end{align}\

\noindent
角速度センサ及び加速度センサからの出力に含まれるノイズ$\delta\omega=[\delta p,\delta q, \delta r]^{T}$及び$\delta a_{b}=[\delta a_{x},\delta a_{y}, \delta a_{z}]^{T}$を考慮すると次式のようになる.

\begin{align}
\dot{\textbf{x}}(t)=\textbf{f}(\textbf{x}(t))+\textbf{B}\textbf{v}
\end{align}\

\noindent
ただし$\textbf{B},\textbf{v}$はそれぞれ


\begin{align} 
\textbf{B}
=
\left[
    \begin{array}{ccc}
      \textbf{0}_{3\times 3}&\textbf{0}_{3\times 3}&\textbf{I}_{3\times 3} \\
      \textbf{I}_{3\times 3}&\textbf{0}_{3\times 3}&\textbf{I}_{3\times 3}
    \end{array}
\right]^{T},
\textbf{v}=[\delta p,\delta q, \delta r,\delta a_{x},\delta a_{y}, \delta a_{z}]^{T}
\end{align}\

\noindent
さらにオイラー法を用いて式(162)を離散化すると

\begin{align}
\textbf{x}(t+1)=\textbf{f}_{t}(\textbf{x}(t))+\textbf{B}_{t}\textbf{v}
\end{align}\

\noindent
ただし

\begin{align}
\textbf{f}_{t}(\textbf{x}(t))=\textbf{x}(t)+\textbf{f}(\textbf{x}(t))\Delta t,\textbf{B}_{t}=\textbf{B}\Delta t
\end{align}\

\noindent
ここで$\Delta t$はサンプリング周期である.

次に観測方程式を求める.今回の位置推定アルゴリズムでは加速度センサと地磁気センサを使って角度の補正を行い,UWBモジュールと距離センサを使って位置の補正及び高度の補正を行う.$\textbf{y}(t)=[a_{x},a_{y},a_{z},m_{x},m_{y},m_{z},d_{1},d_{2},d_{3},d_{h}]^{T}$とおき式(115)~(156)をまとめると次式のようになる.

\begin{align}
\textbf{y}(t)=\textbf{h}_{t}(\textbf{x}(t))
\end{align}\

\noindent
ただし$\textbf{h}(\cdot)$は

\begin{align} 
\textbf{h}_{t}(\textbf{x}(t))
=
\left[
    \begin{array}{c}
       g\sin\theta\\
      -g\cos\theta\sin\phi\\
      -g\cos\theta\cos\phi\\
      m_{x}\cos\theta\cos\psi-m_{z}\sin\theta\\
      m_{x}\sin\phi\sin\theta\cos\psi+m_{z}\cos\theta\sin\phi\\
      m_{x}\cos\phi\sin\theta\cos\psi+m_{z}\cos\theta\cos\phi\\
      \sqrt{(x-x_{1})^{2}+(y-y_{1})^{2}+(z-z_{1})^{2}} \\
      \sqrt{(x-x_{2})^{2}+(y-y_{2})^{2}+(z-z_{2})^{2}} \\
      \sqrt{(x-x_{3})^{2}+(y-y_{3})^{2}+(z-z_{3})^{2}} \\
      \frac{z_{r}}{\cos\theta\cos\phi}
    \end{array}
\right]
\end{align}\

\noindent
式(168)に加速度センサ,地磁気センサ,UWBモジュール及び距離センサから得られるノイズ$[\delta a_{x},\delta a_{y},\delta a_{z}]^{T},[\delta m_{x},\delta m_{y},\delta m_{z}]^{T},[\delta d_{1},\delta d_{2},\delta d_{3}]^{T},\delta d_{h}$を考慮すると観測方程式は

\begin{align}
\textbf{y}(t)=\textbf{h}_{t}(\textbf{x}(t))+\textbf{w}
\end{align}\

\noindent
ただし$\textbf{w}$は

\begin{align}
\textbf{w}=[\delta a_{x},\delta a_{y},\delta a_{z},\delta m_{x},\delta m_{y},\delta m_{z},\delta d_{1},\delta d_{2},\delta d_{3},\delta d_{h}]^{T}
\end{align}\

以上より,式(164)の状態方程式と式(168)の観測方程式より構成されるシステムの離散プロセスモデルが導出された.このプロセスモデルに対して拡張カルマンフィルタを適用することで位置推定アルゴリズムを得ることができる.

\subsection{位置制御}
\subsubsection{位置制御アルゴリズム}

クアッドローターの重心の運動方程式は式(20)より

\begin{align}
m\left[
    \begin{array}{c}
      \ddot{x}^{r}  \\
      \ddot{y}^{r} \\
      \ddot{z}^{r} 
    \end{array}
\right]
=
f_{total}
\left[
    \begin{array}{c}
      \cos\phi\sin\theta\cos\psi + \sin\phi\sin\psi \\
      \cos\phi\sin\theta\sin\psi - \sin\phi\cos\psi \\
      \cos\phi\cos\theta 
    \end{array}
\right]
+m
\left[
    \begin{array}{c}
      0 \\
      0 \\
      -g 
    \end{array}
\right]
\end{align}\

\noindent
ここで$[x_{7},x_{8},x_{9},x_{10},x_{11},x_{12}]^{T}=[x,y,z,\dot{x},\dot{y},\dot{z}]^{T}$とおき,次式のように書き換える.

\begin{align}
\left\{
\begin{aligned}
\dot{x}_{7} &= x_{10}\\
\dot{x}_{10} &= \frac{1}{m}U_{x}f_{total}\\
\dot{x}_{8} &= x_{11}\\
\dot{x}_{11} &= \frac{1}{m}U_{y}f_{total}\\
\dot{x}_{9} &= x_{12}\\
\dot{x}_{12} &= \frac{1}{m}(\cos\phi\cos\theta)f_{total}-g
\end{aligned}
\right.
\end{align}\

\noindent
ただし

\begin{align}
U_{x}=\cos\phi\sin\theta\cos\psi + \sin\phi\sin\psi,U_{y}=\cos\phi\sin\theta\sin\psi - \sin\phi\cos\psi
\end{align}\

\noindent
であり,仮想的な入力とする.まず$x$に関するIntegral Backstepping制御器を導出していく.最初に次式のように変数変換をおこなう.

\begin{align}
e_{7}=x_{d}-x_{7},e_{8}=c_{7}e_{7}+\dot{x}_{d}+\lambda_{4} p_{7} -x_{8}
\end{align}\

\noindent
ただし$x_{d}$は$x$の目標値であり,$c_{7},\lambda_{4}$は正の定数,$p_{7}=\int_0^t e_{7}(\tau)d\tau$である.またリアプノフ関数を次式のように設定する.

\begin{align}
V(e_{7},e_{8})=\frac{1}{2}\lambda_{4} p_{7}^{2}+\frac{1}{2}e_{7}^{2}+\frac{1}{2}e_{8}^{2}
\end{align}

\noindent
また式(174)を時間微分すると

\begin{align}
\dot{V}(e_{7},e_{8})=\lambda_{4} p_{7}\dot{p}_{7}+e_{7}\dot{e}_{7}+e_{8}\dot{e}_{8}
\end{align}\

\noindent
となる.$e_{7}$と$e_{8}$の時間微分はそれぞれ

\begin{align}
\dot{e}_{7}=\dot{x}_{d}-x_{7}=\dot{x}_{d}-(c_{7}e_{7}+\dot{x}_{d}+\lambda_{4} p_{7}-e_{8})=-c_{7}e_{7}-\lambda_{4} p_{7}+e_{8}
\end{align}
\begin{align}
\dot{e}_{8}=c_{7}\dot{e}_{7}+\ddot{x}_{d}+\lambda_{4} e_{1}-\dot{x}_{8}=c_{7}(-c_{7}e_{7}-\lambda_{4} p+e_{8})+\ddot{x}_{d}+\lambda_{4} e_{7}-\frac{U_{x}f_{total}}{m}
\end{align}\

\noindent
となる.ここで仮想的な入力$U_{x}$を次式のように設定する.

\begin{align}
U_{x}=\frac{m}{f_{total}}((1-c_{7}^{2}-\lambda_{4})e_{7}+(c_{7}+c_{8})e_{8}+\lambda_{7} p_{7}c_{7}+\ddot{x}_{d})
\end{align}\

\noindent
ただし,$c_{8}$は正の定数である.したがって$V(e_{7},e_{8})\leq 0$となりシステムは安定になる.式(178)に式(173)を代入して$x$で表すと

\begin{align}
U_{x}=\frac{m}{f_{total}}((1+c_{7}c_{8}+\lambda_{4})(x_{d}-x)+(c_{7}+c_{8})(\dot{x}_{d}-\dot{x})+c_{8}\lambda_{4}\int_0^t (x_{d}-x)d\tau)
\end{align}\

\noindent
となる.ただし$\ddot{x}_{d}=0$とおいた.$y$に関しても同様に計算すると

\begin{align}
U_{y}=\frac{m}{f_{total}}((1+c_{9}c_{10}+\lambda_{5})(y_{d}-y)+(c_{9}+c_{10})(\dot{y}_{d}-\dot{y})+c_{10}\lambda_{5}\int_0^t (y_{d}-y)d\tau)
\end{align}\

\noindent
ただし$c_{9},c_{10},\lambda_{5}$は正の定数であり,$y_{d}$は$y$の目標値である.次に$U_{x},U_{y}$から目標姿勢角$\phi_{d},\theta_{d}$を算出する.式(172)を$\phi,\theta$について解き$\phi=\phi_{d},\theta=\theta_{d},\phi=\psi_{d}$を代入すると

\begin{align}
\left\{
\begin{aligned}
\phi_{d} &= \sin^{-1}(U_{x}\sin\psi_{d}-U_{y}\cos\psi_{d})\\
\theta_{d} &= \sin^{-1}(\frac{1}{\cos\phi_{d}}(U_{x}\cos\psi_{d}+U_{y}\sin\psi_{d}))
\end{aligned}
\right.
\end{align}\

\noindent
となる.以上より目標姿勢角$\phi_{d},\theta_{d}$を求めることができた.


次に$z$に関するIntegral Backstepping制御器を導出していく.最初に次式のように変数変換をおこなう.

\begin{align}
e_{11}=z_{d}-x_{11},e_{12}=c_{11}e_{11}+\dot{z}_{d}+\lambda_{6} p_{11} -x_{12}
\end{align}\

\noindent
ただし$z_{d}$は$z$の目標値であり,$c_{11},\lambda_{6}$は正の定数,$p_{11}=\int_0^t e_{11}(\tau)d\tau$である.またリアプノフ関数を次式のように設定する.

\begin{align}
V(e_{11},e_{12})=\frac{1}{2}\lambda_{6} p_{11}^{2}+\frac{1}{2}e_{11}^{2}+\frac{1}{2}e_{12}^{2}
\end{align}

\noindent
また式(183)を時間微分すると

\begin{align}
\dot{V}(e_{11},e_{12})=\lambda_{6} p_{11}\dot{p}_{11}+e_{11}\dot{e}_{11}+e_{12}\dot{e}_{12}
\end{align}\

\noindent
となる.$e_{11}$と$e_{12}$の時間微分はそれぞれ

\begin{align}
\dot{e}_{11}=\dot{z}_{d}-x_{11}=\dot{z}_{d}-(c_{11}e_{11}+\dot{z}_{d}+\lambda_{6} p_{11}-e_{12})=-c_{11}e_{11}-\lambda_{6} p+e_{12}
\end{align}
\begin{align}
\dot{e}_{12}=c_{11}\dot{e}_{11}+\ddot{z}_{d}+\lambda_{6} e_{11}-\dot{x}_{12}=c_{11}(-c_{11}e_{11}-\lambda_{6} p+e_{12})+\ddot{z}_{d}+\lambda_{6} e_{11}-\frac{1}{m}(\cos\phi\cos\theta)f_{total}+g
\end{align}\

\noindent
となる.ここ全推力$f_{total}$を次式のように設定する.

\begin{align}
f_{total}=\frac{m}{\cos\phi\cos\theta}((1-c_{11}^{2}-\lambda_{6})e_{11}+(c_{11}+c_{12})e_{12}+\lambda_{6} p_{11}c_{11}+\ddot{\phi}_{d}+g)
\end{align}\

\noindent
ただし,$c_{12}$は正の定数である.したがって$V(e_{11},e_{12})\leq 0$となりシステムは安定になる.式(187)に式(182)を代入して$z$で表すと

\begin{align}
f_{total}=\frac{1}{\cos\phi\cos\theta}((1+c_{11}c_{12}+\lambda_{6})(z_{d}-z)+(c_{11}+c_{12})(\dot{z}_{d}-\dot{z})+c_{12}\lambda_{6}\int_0^t (z_{d}-z)d\tau+g)
\end{align}\

\noindent
となる.

\newpage
\section{位置推定アルゴリズムの評価}

7.1で述べた拡張カルマンフィルタを用いた位置推定アルゴリズムの評価を行った.評価方法としては,位置推定アルゴリズムによって得られたクアッドロータの位置とステレオカメラシステムで得られた位置を比較した.

Fig. 80に実験装置の概略図を示す.4つのUWBの基地局をクアッドロータの周りに配置してある.各基地局の座標をTable 7に示す.またクアッドロータには赤外線LEDマーカーが搭載されており,ステレオカメラシステムでもクアッドロータの位置の測定が可能である.

\begin{figure}[H]
\begin{center}
\includegraphics[height=70mm]{./Fig/Schematic_diagram_of_experimental_apparatus_for_position_estimation.png}
\caption{Schematic diagram of experimental apparatus for position estimation.}
\label{Schematic_diagram_of_experimental_apparatus_for_position_estimation}
\end{center}
\end{figure}

\begin{table}[H]
 \caption{Absolute position of anchors.}
 \label{Absolute_position_of_anchors}
 \begin{center}
  \begin{tabular}{|c|c|c|c|}
   \hline
      & $x$ position [m] & $y$ position [m] & $z$ position [m] \\
   \hline\hline
   Anchor1& -6.000 & 3.500 & 2.500 \\
   \hline 
   Anchor2&  8.695 & 3.514 & 2.500 \\
   \hline
   Anchor3&  8.436 & -8.166 & 2.500 \\
   \hline
   Anchor4& -6.003 & -8.160 & 2.500 \\
   \hline
  \end{tabular}
 \end{center}
\end{table}

\noindent
位置推定アルゴリズムはクアッドロータに搭載されているシングルボードコンピュータで行い,推定結果はWi-Fi経由でPCに送信し記録する.またクアッドロータとステレオカメラシステムとの同期は赤外線LEDマーカーの点灯した瞬間を初期時刻として同期を行う.位置推定アルゴリズムのサンプリング周波数とステレオカメラシステムのサンプリング周波数に関しては100 Hzに設定し実験を行った.

次にクアッドロータの動かし方について説明する.クアッドロータは手で持って,Fig. 81に示す経路を通るように動かす.各ポイントP1からP8の座標をTable 8に示す.

\begin{figure}[H]
\begin{center}
\includegraphics[height=80mm]{./Fig/Moving_path_of_the_quad-rotor.png}
\caption{Moving path of the quad-rotor.}
\label{Moving_path_of_the_quad-rotor}
\end{center}
\end{figure}

\begin{table}[H]
 \caption{Waypoint location.}
 \label{Waypoint_location}
 \begin{center}
  \begin{tabular}{|c|c|c|c|}
   \hline
      & $x$ position [m] & $y$ position [m] & $z$ position [m] \\
   \hline\hline
   P1& 0.00 & 0.00 & 0.00 \\
   \hline 
   P2& 0.00 & 0.00 & 1.00 \\
   \hline
   P3& 0.00 & -1.25 & 1.00 \\
   \hline
   P4& 2.50 & -1.25 & 1.00 \\
   \hline
   P5& 2.50 & 1.25 & 1.00 \\
   \hline
   P6& 0.00 & 1.25 & 1.00 \\
   \hline
   P7& 0.00 & 0.00 & 1.00 \\
   \hline
   P8& 0.00 & 0.00 & 0.00 \\
   \hline
  \end{tabular}
 \end{center}
\end{table}


今回UWBの基地局の数と位置推定の精度を調査するため,基地局が3つの場合と4つの場合でそれぞれ5回ずつ位置推定を行った.Figs. 82~91に位置推定アルゴリズムによる位置推定結果とステレオカメラシステムによる位置測定結果を示す.

\newpage

\begin{figure}[H]
\begin{center}
\includegraphics[height=210mm]{./Fig/Quadrotor_position_estimation_result_with_3_anchors_and_measurement_result_by_stereo_camera_system(test1).png}
\caption{Quad-rotor position estimation result with 3 anchors and measurement result by stereo camera system (test1).}
\label{Quadrotor_position_estimation_result_with_3_anchors_and_measurement_result_by_stereo_camera_system(test1)}
\end{center}
\end{figure}
\newpage

\begin{figure}[H]
\begin{center}
\includegraphics[height=210mm]{./Fig/Quadrotor_position_estimation_result_with_3_anchors_and_measurement_result_by_stereo_camera_system(test2).png}
\caption{Quad-rotor position estimation result with 3 anchors and measurement result by stereo camera system (test2).}
\label{Quadrotor_position_estimation_result_with_3_anchors_and_measurement_result_by_stereo_camera_system(test2)}
\end{center}
\end{figure}
\newpage

\begin{figure}[H]
\begin{center}
\includegraphics[height=210mm]{./Fig/Quadrotor_position_estimation_result_with_3_anchors_and_measurement_result_by_stereo_camera_system(test3).png}
\caption{Quad-rotor position estimation result with 3 anchors and measurement result by stereo camera system (test3).}
\label{Quadrotor_position_estimation_result_with_3_anchors_and_measurement_result_by_stereo_camera_system(test3)}
\end{center}
\end{figure}
\newpage

\begin{figure}[H]
\begin{center}
\includegraphics[height=210mm]{./Fig/Quadrotor_position_estimation_result_with_3_anchors_and_measurement_result_by_stereo_camera_system(test4).png}
\caption{Quad-rotor position estimation result with 3 anchors and measurement result by stereo camera system (test4).}
\label{Quadrotor_position_estimation_result_with_3_anchors_and_measurement_result_by_stereo_camera_system(test4)}
\end{center}
\end{figure}
\newpage

\begin{figure}[H]
\begin{center}
\includegraphics[height=210mm]{./Fig/Quadrotor_position_estimation_result_with_3_anchors_and_measurement_result_by_stereo_camera_system(test5).png}
\caption{Quad-rotor position estimation result with 3 anchors and measurement result by stereo camera system (test5).}
\label{Quadrotor_position_estimation_result_with_3_anchors_and_measurement_result_by_stereo_camera_system(test5)}
\end{center}
\end{figure}
\newpage

\begin{figure}[H]
\begin{center}
\includegraphics[height=210mm]{./Fig/Quadrotor_position_estimation_result_with_4_anchors_and_measurement_result_by_stereo_camera_system(test1).png}
\caption{Quad-rotor position estimation result with 4 anchors and measurement result by stereo camera system (test1).}
\label{Quadrotor_position_estimation_result_with_4_anchors_and_measurement_result_by_stereo_camera_system(test1)}
\end{center}
\end{figure}
\newpage

\begin{figure}[H]
\begin{center}
\includegraphics[height=210mm]{./Fig/Quadrotor_position_estimation_result_with_4_anchors_and_measurement_result_by_stereo_camera_system(test2).png}
\caption{Quad-rotor position estimation result with 4 anchors and measurement result by stereo camera system (test2).}
\label{Quadrotor_position_estimation_result_with_4_anchors_and_measurement_result_by_stereo_camera_system(test2)}
\end{center}
\end{figure}
\newpage

\begin{figure}[H]
\begin{center}
\includegraphics[height=210mm]{./Fig/Quadrotor_position_estimation_result_with_4_anchors_and_measurement_result_by_stereo_camera_system(test3).png}
\caption{Quad-rotor position estimation result with 4 anchors and measurement result by stereo camera system (test3).}
\label{Quadrotor_position_estimation_result_with_4_anchors_and_measurement_result_by_stereo_camera_system(test3)}
\end{center}
\end{figure}
\newpage

\begin{figure}[H]
\begin{center}
\includegraphics[height=210mm]{./Fig/Quadrotor_position_estimation_result_with_4_anchors_and_measurement_result_by_stereo_camera_system(test4).png}
\caption{Quad-rotor position estimation result with 4 anchors and measurement result by stereo camera system (test4).}
\label{Quadrotor_position_estimation_result_with_4_anchors_and_measurement_result_by_stereo_camera_system(test4)}
\end{center}
\end{figure}
\newpage

\begin{figure}[H]
\begin{center}
\includegraphics[height=210mm]{./Fig/Quadrotor_position_estimation_result_with_4_anchors_and_measurement_result_by_stereo_camera_system(test5).png}
\caption{Quad-rotor position estimation result with 4 anchors and measurement result by stereo camera system (test5).}
\label{Quadrotor_position_estimation_result_with_4_anchors_and_measurement_result_by_stereo_camera_system(test5)}
\end{center}
\end{figure}
\newpage




また実験結果から各座標$x,y,z$におけるRoot Mean Squared Error (RMSE)を算出した.位置推定アルゴリズムによる推定位置を$[\hat{x}_{i},\hat{y}_{i},\hat{z}_{i}]^{T}(i=1,...,n)$,ステレオカメラシステムによる位置測定結果を$[x_{camera},y_{camera},z_{camera}]^{T}(i=1,...,n)$とおいた時,各座標のRMSEの定義式は以下のようになる.ただし$n$はデータ数とする.

\begin{align}
RMSE_{x}=\sqrt{\frac{1}{n}\sum_{i=1}^{n}(\hat{x}_{i}-x_{camera})^{2}}
\end{align}

\begin{align}
RMSE_{y}=\sqrt{\frac{1}{n}\sum_{i=1}^{n}(\hat{y}_{i}-y_{camera})^{2}}
\end{align}

\begin{align}
RMSE_{z}=\sqrt{\frac{1}{n}\sum_{i=1}^{n}(\hat{z}_{i}-z_{camera})^{2}}
\end{align}\

\noindent
式(189)~(191)を用いて算出した各座標のRMSEをTable 9とTable 10に示す.


\begin{table}[H]
 \caption{RMSE of each coordinate and average (3 anchors).}
 \label{RMSE3}
 \begin{center}
  \begin{tabular}{|c|c|c|c|}
   \hline
      & $x$ position [m] & $y$ position [m] & $z$ position [m] \\
   \hline\hline
   Test1& 0.207 & 0.207 & 0.051 \\
   \hline 
   Test2& 0.224 & 0.263 & 0.061 \\
   \hline
   Test3& 0.212 & 0.180 & 0.074 \\
   \hline
   Test4& 0.210 & 0.223 & 0.086 \\
   \hline
   Test5& 0.225 & 0.208 & 0.085 \\
   \hline\hline
   Average& 0.216 & 0.216 & 0.071 \\
   \hline
  \end{tabular}
 \end{center}
\end{table}

\begin{table}[H]
 \caption{RMSE of each coordinate and average (4 anchors).}
 \label{RMSE4}
 \begin{center}
  \begin{tabular}{|c|c|c|c|}
   \hline
      & $x$ position [m] & $y$ position [m] & $z$ position [m] \\
   \hline\hline
   Test1& 0.126 & 0.125 & 0.089 \\
   \hline 
   Test2& 0.180 & 0.153 & 0.094 \\
   \hline
   Test3& 0.158 & 0.150 & 0.078 \\
   \hline
   Test4& 0.163 & 0.154 & 0.085 \\
   \hline
   Test5& 0.160 & 0.150 & 0.076 \\
   \hline\hline
   Average& 0.158 & 0.146 & 0.084 \\
   \hline
  \end{tabular}
 \end{center}
\end{table}

Figs. 82~91の$x$及び$y$の推定値と$z$の推定値の時系列データを比較すると$z$の推定値の方が,$x,y$の推定値に比べ精度よく推定できていることがわかる.それぞれのRMSEを比較してもTable 9,Table 10からわかるように$x,y$のRMSEに比べ$z$のRMSEが小さくなっている.これは機体の高度をUWBだけでなく距離センサでも測定しているからである.また基地局が3つの場合と4つの場合のクアッドロータの推定結果を比較すると,4つの場合の方が推定値の突発的な変化が3つの場合に比べ減少していることがわかる.これはクアッドロータの位置を観測するセンサが増えたことにより,位置の推定精度が上がったためと考えられる.このことはTable 9,Table 10のRMSEの比較からもわかり,$x,y$のRMSEが基地局が3つの場合と4つの場合でおよそ0.05 mの差があることがわかる.

以上から距離センサの追加と基地局の増設により,クワッドロータの位置推定の精度が向上することがわかった.


\newpage
\section{位置制御アルゴリズムの評価}

位置の制御性能を評価する最初の段階として,水平方向のみIntegral Backstepping制御を適用し,スロットルのみ操縦者がリモコンを介して操作してホバリングさせる実験を行った.$x,y$の目標値はそれぞれ0である.また実験中のクアッドロータの位置はステレオカメラシステムで計測している.位置制御ループの周波数は100 Hzで行った.実験の様子をFig. 92に示す.クアッドロータは墜落を防止するため,上から垂らしているひもにくくりつけてある.ひもは常にたるんだ状態にしてあり,クアッドロータの位置制御には影響を及ぼさない.

\begin{figure}[H]
\begin{center}
\includegraphics[height=150mm]{./Fig/Position_control_experiment.png}
\caption{Position control experiment.}
\label{Position_control_experiment}
\end{center}
\end{figure}

位置制御中のドローンの動きをみると目標に追従しようとしているが,4 secあたりから$y$方向の動きが振動的になってしまい目標値から徐々に離れてしまっている.そして9 secあたりで足が地面に接触し,制御不能の状態に陥ってしまった.原因を調べるために振動的になっているクアッドロータのy軸の位置とピッチ角に着目した.実験中における$y$軸の位置とピッチ角の時系列データをFig. 93とFig. 94に示す.Fig. 93,Fig. 94を比較すると位置の動きに対してピッチ角の制御が遅れていることがわかる.したがって今回クアッドロータの姿勢角の応答性が悪いため,位置制御が振動的になってた考えられる.

\begin{figure}[H]
\begin{center}
\includegraphics[height=70mm]{./Fig/Quad-rotor_position_during_position_control.png}
\caption{Quad-rotor position during position control.}
\label{Quad-rotor_position_during_position_control}
\end{center}
\end{figure}

\begin{figure}[H]
\begin{center}
\includegraphics[height=70mm]{./Fig/Quad-rotor_attitude_angle_during_position_control.png}
\caption{Quad-rotor attitude angle during position control.}
\label{Quad-rotor_attitude_angle_during_position_control}
\end{center}
\end{figure}





\newpage
%%%%%%%%%%%%%%%%%%%%%%%%%%%%%%%%%%%%%%%%%%%%%%%%%%%%%%%%%%%%%%%%%%%%%%%%%%%%%
%%%%%%%%%%%%%%%%%%%%%% 結言 %%%%%%%%%%%%%%%
%%%%%%%%%%%%%%%%%%%%%%%%%%%%%%%%%%%%%%%%%%%%%%%%%%%%%%%%%%%%%%%%%%%%%%%%%%%%%%%%%%%%%%%%%%
\section{結言}
本研究では鋼構造物点検用UAVの非GPS環境での自律飛行を目的として,UWB,IMU及び距離センサを組み合わせた自己位置推定法及びIntegral Backstepping制御を使った自律飛行制御手法を提案し,制御器の構築,小型クアッドロータを用いた実機開発を行った.

姿勢推定に関しては加速度センサ,ジャイロセンサ,地磁気センサを拡張カルマンフィルタにより組み合わせた手法を提案した.また地磁気センサのキャリブレーション方法についても説明を行った.ポテンショメータとの比較では,ポテンショメータからの角度と推定値に若干の位相遅れがあるものの高い精度で姿勢角が推定できていることを示すことができた.また推定に地磁気センサを使っているヨー角に関しては,ロータの回転による磁気の影響を調べるためにロータを回転させた状態で角度の推定を行った.回転させてない時と比べ偏差は大きくなったものの,大きくずれることはなくクアッドロータの飛行中にもヨー角が推定可能であることを示した.

姿勢制御に関してはIntegral Backstepping制御を用いた姿勢制御器の構築し,その姿勢制御器を評価のために,ロール角,ピッチ角,ヨー角それぞれ単独で評価を行った結果,どの姿勢角も目標値に対して追従できることが示された.また,ピッチ角,ロール角,ヨー角を組み合わせた制御性能を評価するためにクアッドロータのホバリング実験を行った.実験ではスロットルの操作のみで,ホバリングができることが示された.またその時のクアッドロータの飛行経路も示した.

位置推定に関しては9軸のIMU,UWB,距離センサを拡張カルマンフィルタにより組み合わせた手法を提案し,推定した位置とステレオカメラにより測定した位置との比較を行った.その結果,提案した手法を用いることで,非GPS環境においてクアッドロータの水平方向に関してはおよそ0.15 m,高度に関しては0.08 mの精度で推定できることが示された.また,距離センサの追加によって高度の推定精度が向上することが示された.さらに基地局を3つから4つに増やすことで,水平の位置精度がおよそ28%向上することが示された.

位置制御に関しては逆ダイナミクス法とIntegral Backstepping制御を組み合わせた位置制御器の構築を行った.位置制御器を評価する最初の段階として,水平方向のみ位置制御を適用しスロットルのみ操作してクアッドロータをホバリングさせる実験を行ったが,クアッドロータの$y$方向の動きが振動的になり,目標に追従させることができなかった.原因としては姿勢制御の応答性が悪く,位置制御が振動的になったためであると考えられる.現状では,応答性を上げるために姿勢制御の制御ゲインを上げると,発散するため,姿勢角が目標値に対して収束しない.そのため姿勢制御の応答性を向上させるためには,姿勢制御ループの周波数を現在の100 Hzから引き上げる必要があると考えられる.






\newpage
%%%%%%%%%%%%%%%%%%%%%%%%%%%%%%%%%%%%%%%%%%%%%%%%%%%%%%%%%%%%%%%%%%%%%%%%%%%%%%%%%%%%%%%%%%
%%%%%%%%%%%%%%%%%%%%%% 謝辞 %%%%%%%%%%%%%%%
%%%%%%%%%%%%%%%%%%%%%%%%%%%%%%%%%%%%%%%%%%%%%%%%%%%%%%%%%%%%%%%%%%%%%%%%%%%%%%%%%%%%%%%%%%
\section*{謝辞}
\addcontentsline{toc}{section}{謝辞}


この研究を行う機会を下さった澤田祐一教授,東善之助教に感謝致します.また,報告会や勉強会を通して多くの知識や示唆をくださったロボティクス研究室の皆様及び学系プロジェクトの皆様に感謝致します.加えて,機体製作に当たり協力していただきましたものづくり教育研究支援センターの方々にも心より感謝致します.


また,本研究の一部は一般財団法人日本建設情報総合センターの支援により行われました.ここに謝意を表します.


\newpage
%%%%%%%%%%%%%%%%%%%%%%%%%%%%%%%%%%%%%%%%%%%%%%%%%%%%%%%%%%%%%%%%%%%%%%%%%%%%%%%%%%%%%%%%%%
%%%%%%%%%%%%%%%%%%%%%%%%%%%%%%%%%%%%%%%%%%%%%%%%%%%%%%%%%%%%%%%%%%%%%%%%%%%%%%%%%%%%%%%%%%
%%%%%%%%%%%%%%%%%%%%%%%%%%%%%%%%%%%%%%%%%%%%%%%%%%%%%%%%%%%%%%%%%%%%%%%%%%%%%%%%%%%%%%%%%%
\begin{thebibliography}{99}
\addcontentsline{toc}{section}{参考文献}


\bibitem{a}
国土交通省,橋梁の現状,\\
\url{http://www.mlit.go.jp/sogoseisaku/maintenance/_pdf/research01_pdf01.pdf},2018/02/02閲覧

\bibitem{p}
国土交通省,道路構造物の現状(橋梁),\\
\url{http://www.mlit.go.jp/road/sisaku/yobohozen/yobo1_1.pdf},2018/02/02閲覧

\bibitem{b}
大阪市立大学,橋の鋼鉄部材を検査できる橋梁検査ロボット バイリム(BIREM)の開発,\\
\url{https://www.osaka-cu.ac.jp/ja/news/2013/e4oxuo},2018/02/02閲覧

\bibitem{c}
次世代インフラ用ロボット技術・ロボットシステム~現場実証ポータルサイト~,\\
\url{http://www.c-robotech.info/}平成26年度現場検証技術\url{db-1/}平成26年度橋梁維持管理部会\url{db/}橋梁-富士フイルム,2018/02/02閲覧

\bibitem{d}
水谷将馬ら,錯雑した構造体中で飛行が可能な回転球殻を持つクアッドロータ,ロボティクス・メカトロニクス講演会2014,(2014),1-4.

\bibitem{e}
東北大学,Human-Robot Informatics Laboratory, ぶつかっても落ちないドローン(球殻ヘリ)の研究,\\
\url{http://www.mlit.go.jp/sogoseisaku/maintenance/_pdf/research01_pdf01.pdf},2018/02/02閲覧

\bibitem{f}
構造物メンテナンス研究センター CAESAR,\\
\url{http://www.pwri.go.jp/caesar/manual/},2018/02/02閲覧

\bibitem{g}
S. Akahori, Y. Higashi, A. Masuda, K. Takeuchi,
Development of a Vibration Probe Foot Using an EPM for Aerial Inspection Robots, Proc. Of the 10th International Symposium on Advanced Science and Technology in Experimental Mechanics, (2015), 55.

\bibitem{h}
S. Akahori, Y. Higashi, A. Masuda,
Development of an Aerial Inspection Robot with EPM and Camera Arm for Steel Structures, TENCON 2016 - 2016 IEEE Region 10 Conference,
(2016), 1264.

\bibitem{i}
Autonomous Control Systems Laboratory Ltd.,\\
\url{http://www.acsl.co.jp/products/},2018/02/02閲覧

\bibitem{j}
decaWave,\\
\url{https://www.decawave.com/products/dwm1000-module},2018/02/02閲覧

\bibitem{k}
G. Bradski, A. Kaehler, 松田晃一,詳解OpenCVコンピュータビジョンライブラリを使った画像処理・認識,(2012),377-387,423-432,オーム社.

\bibitem{l}
M. Kok, T. B. Schon, Magnetometer Calibration Using Inertial Sensors, IEEE Sensors Journal Volume: 16, (2016), 5679-5689.

\bibitem{m}
Q. Li, J. G. Griffiths, Least Squares Ellipsoid Specific Fitting, Geometric Modeling and Processing 2014, (2014), 2-5.

\bibitem{n}
足立修一,丸田一郎,カルマンフィルタの基礎,(2013),99-110,東京電機大学出版局.

\bibitem{o}
T. Madani, A. Benallegue, Backstepping Control for a Quadrotor Helicopter, Proceedings of the 2006 IEEE/RSJ International Conference on Intelligent Robots and System, 9419317, (2006), 1-6.

\bibitem{q}
Z. Fang, W. Gao, Adaptive Integral Backstepping Control of a Micro-Quadrotor, Intelligent Control and Information Processing (ICICIP), 12217869, (2011), 1-6.

\bibitem{r}
Z. Zuo, Quadrotor Trajecotry Tracking Control: A PD Control Algorithm, 2010 3rd International Conference on Computer and Electrical Engineering (ICCEE 2010), (2010), 3-4.


\end{thebibliography}


\end{document}